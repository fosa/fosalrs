% coding:utf-8

%FOSALRS, a LaTeX-Code for a electrical summary of control theory
%Copyright (C) 2013, Daniel Winz, Ervin Mazlagic

%This program is free software; you can redistribute it and/or
%modify it under the terms of the GNU General Public License
%as published by the Free Software Foundation; either version 2
%of the License, or (at your option) any later version.

%This program is distributed in the hope that it will be useful,
%but WITHOUT ANY WARRANTY; without even the implied warranty of
%MERCHANTABILITY or FITNESS FOR A PARTICULAR PURPOSE.  See the
%GNU General Public License for more details.
%----------------------------------------

\chapter{Laplace Transformation}

\section{Laplace-Transformation}
Sei $u(t)$ ein Signal. \\
Falls das Integral 
\[ U(s) = \int\limits_{0}^{\infty} u(t) \cdot e^{-st} ~ dt \]
konvergiert, so heisst 
\[ \boxed{\mathcal{L} \lbrace u(t) \rbrace = U(s)} \]
die Laplace-Transformierte von $u(t)$. \\
Erklärungen: 
\begin{itemize}
  \item $s \in \mathbb{C}$
  \item $\int\limits_{0^-}^{\infty} u(t) \cdot e^{-st} ~ dt 
  = \lim_{a \nearrow 0} \int\limits_{a}^{\infty} u(t) \cdot e^{-st} ~ dt $
  \item Falls $u(t)$ sprungstetig in $t = 0$ ist, kann man$0^-$ durch $0$ 
        ersetzen. \\
        $u(t)$ heisst sprungstetig, wenn es nur endliche Sprünge macht. 
  \item Wenn $u(t)$ sprungstetig ist und höchstens exponentiell wächst, 
        dann konvergiert das Integral für alle $s \in \mathbb{C}$ mit 
        $\text{Re}(s) > \sigma_0$\\
        % >>> Grafik Konvergenzbereich
\end{itemize}
\[ s = \sigma + \omega \cdot j, \qquad Re(s) = \sigma 
\qquad \sigma, \omega \in \mathbb{R} \]
Die Menge 
\[ KB = \lbrace s \in \mathbb{C} | (*) konvergiert \rbrace 
= \lbrace s \in \mathbb{R} | \text{Re}(s) > \sigma_0 \rbrace \]
heisst Konvergenzbereich. \\
Bsp: 
\[ u(t) = \sigma(t) \]
\[ \mathcal{L} \lbrace \sigma(t) \rbrace 
= \int\limits_{0^-}^{\infty} \underbrace{\sigma(t) \cdot e^{-st}}_
{\text{sprungstetig in }t = 0} ~ dt = \int\limits_{0}^{\infty} e^{-st} ~ dt 
= \left. -\frac{1}{s} e^{-st} \right|_{t = 0}^{t = \infty} 
= -\frac{1}{s} \underbrace{\lim_{t \to \infty} \cdot e^{-st}}_
{0 \text{ falls Re}(s) > 0} + \frac{1}{s} \]
Also
\[ \mathcal{L} \lbrace \sigma(t) \rbrace = \frac{1}{s} \]
Andere Notation: 
\[ \sigma(t) ~ \laplace ~ \frac{1}{s} \qquad \text{KB: Re}(s) > 0\]




\[ \boxed{\mathcal{L} \lbrace u(t) \rbrace 
= \int\limits_{0}^{\infty} u(t) \cdot e^{-st} ~ dt \qquad , s \in \mathbb{C} 
= U(s)} \]
\[ \boxed{\begin{array}{rcl}
\text{Zeitbereich} &  & \text{Bildbereich} \\
u(t) & \laplace & U(s)
\end{array}} \]


\subsection{Eigenschaften der Laplace-Transformation}

\subsubsection{Linearität}
\[ \boxed{\mathcal{L}\lbrace c \cdot u(t) \rbrace 
= c \cdot \mathcal{L}\lbrace u(t) \rbrace} \]
\[ \boxed{\mathcal{L}\lbrace u_1(t) \cdot u2(t) \rbrace 
= \mathcal{L}\lbrace u1(t) \rbrace + \mathcal{L}\lbrace u1(t) \rbrace} \]

\subsubsection{Differentiationssatz}
\[ \boxed{\mathcal{L}\lbrace \dot{u}(t) \rbrace = s \cdot U(s) - u(0^-)} \]
\[ \boxed{\dot{u}(t) ~ \laplace ~ s \cdot U(s) - u(0^-)} \]
\[ \boxed{\dddot{u}(t) ~ \laplace ~ s^3 \cdot U(s) - s^2 \cdot u(0^-) 
- s \cdot \dot{u}(0^-) - \ddot{u}(0^-)} \]
Häufig: Anfangsbedingungen $= 0$. 
Dann: 
\[ \boxed{\frac{d^n}{dt^n} u(t) ~ \laplace ~ s^n \cdot U(S)} \]

\subsection{Integrationssatz}
\[ \boxed{\int\limits_{0^-}^{t} u(t) ~ dt ~ \laplace ~ \frac{1}{s} \cdot U(s)} \]

\subsection{Zeitverschiebungssatz}
Ein Signal u(t) heisst kausal, wenn $u(t) = 0$ für $t < 0$. 
\[ \boxed{u(t - t_0) \cdot \sigma(t - t_0) ~ \laplace ~ e^{-t_0s} \cdot U(s)} \]

\subsection{Laplace Transformation einiger Grundfunktionen}
\[ \begin{array}{ccc}
u(t) & \laplace & U(s) \\\\
\delta(t) & 
    & 1 \\\\
\sigma(t) & 
    & \frac{1}{s} \\\\
t \cdot \sigma(t) & 
    & \frac{1}{s^2} \\\\
t^2 \cdot \sigma(t) & 
    & \frac{2}{s^3} \\\\
\vdots & 
    & \vdots \\\\
e^{a \cdot t} & 
    & \frac{1}{s - a} \\\\
\sin(t) & 
    & \frac{1}{s^2 + 1} \\\\
\cos(t) & 
    & \frac{s}{s^2 + 1} \\\\
\end{array} \]
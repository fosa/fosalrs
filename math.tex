% coding:utf-8

%FOSALRS, a LaTeX-Code for a electrical summary of control theory
%Copyright (C) 2013, Daniel Winz, Ervin Mazlagic

%This program is free software; you can redistribute it and/or
%modify it under the terms of the GNU General Public License
%as published by the Free Software Foundation; either version 2
%of the License, or (at your option) any later version.

%This program is distributed in the hope that it will be useful,
%but WITHOUT ANY WARRANTY; without even the implied warranty of
%MERCHANTABILITY or FITNESS FOR A PARTICULAR PURPOSE.  See the
%GNU General Public License for more details.
%----------------------------------------

\section{Signale und Systeme}
% \emph{Steuerung: }Vorgang zur Beeinflussung einer Grösse \\
% \emph{Regelung: }Steuerung + Messung + Rückkopplung \\
% Nichtlinearität + Rückkopplung = Chaos \\
% $\rightarrow$ Linearität wichtig \\
% Typische Steuerung: \\
% >>>FOSA: Bild 1\\
% Blöcke sind Systeme: \\
% System: \\
% Gesetzmässigkeit, mit der eine Grösse u(t) eine Grösse v(t) beeinflusst. \\
% ("`u steuert v"')\\
% >>>FOSA: Bild 2

\subsection{Mathematische Modellierung}
\emph{Signal: }Funktion $u(t)$ der Zeit $t$\\
Bsp: 
\begin{itemize}
  \item Konstantes Signal: \\
        $u(t) = u_0$ \\
        >>>FOSA: Bild Konstantes Signal
  \item Harmonische Schwingungen: \\
        $u(t) = A \cdot \cos(\omega t + \varphi)$ \\
        >>>FOSA: Bild Sinussignal
  \item Einheitssprung \\
        $\sigma(t) = 
        \left\lbrace 
        \begin{array}{l}
        0, t<0 \\
        1, t \geq 0
        \end{array} 
        \right. $\\
        >>>FOSA: Bild Einheitssprung
\end{itemize}

\subsection{Ableitung an einer Sprungstelle}
Grundsignal: Einheitssprung\\
\[ \boxed{\begin{array}{ll}
\frac{d \sigma}{d t} = \dot{\sigma}(t) = 0 & \forall t \neq 0 \\
\frac{d \sigma}{d t} = \dot{\sigma}(t) = \infty & | t =   0
\end{array}
} \]
\[ \boxed{\dot{\sigma}(t) = \left\lbrace 
\begin{array}{ll}0&, t \neq 0\\\infty&, t = 0\end{array} \right.} \]
Diese Funktion heisst Diracsche Deltafunktion, Einheitsstoss, Diracstoss\\
Notation: $\delta(t)$
\[ \boxed{\dot{\sigma}(t) = \delta(t)} \]
1. Methode: \\
Linearität der Ableitung\\\\
2. Methode: \\
Graphisch\\
>>>FOSA: Bild für grafische Ableitung
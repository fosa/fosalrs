% coding:utf-8

%FOSALRS, a LaTeX-Code for a electrical summary of control theory
%Copyright (C) 2013, Daniel Winz, Ervin Mazlagic

%This program is free software; you can redistribute it and/or
%modify it under the terms of the GNU General Public License
%as published by the Free Software Foundation; either version 2
%of the License, or (at your option) any later version.

%This program is distributed in the hope that it will be useful,
%but WITHOUT ANY WARRANTY; without even the implied warranty of
%MERCHANTABILITY or FITNESS FOR A PARTICULAR PURPOSE.  See the
%GNU General Public License for more details.
%----------------------------------------

\section{Signale und Systeme}
% \emph{Steuerung: }Vorgang zur Beeinflussung einer Grösse \\
% \emph{Regelung: }Steuerung + Messung + Rückkopplung \\
% Nichtlinearität + Rückkopplung = Chaos \\
% $\rightarrow$ Linearität wichtig \\
% Typische Steuerung: \\
% >>>FOSA: Bild 1\\
% Blöcke sind Systeme: \\
% System: \\
% Gesetzmässigkeit, mit der eine Grösse u(t) eine Grösse v(t) beeinflusst. \\
% ("`u steuert v"')\\
% >>>FOSA: Bild 2

\subsection{Mathematische Modellierung}
\emph{Signal: }Funktion $u(t)$ der Zeit $t$\\
Bsp: 
\begin{itemize}
  \item Konstantes Signal: \\
        $u(t) = u_0$ \\
        >>>FOSA: Bild Konstantes Signal
  \item Harmonische Schwingungen: \\
        $u(t) = A \cdot \cos(\omega t + \varphi)$ \\
        >>>FOSA: Bild Sinussignal
  \item Einheitssprung \\
        $\sigma(t) = 
        \left\lbrace 
        \begin{array}{l}
        0, t<0 \\
        1, t \geq 0
        \end{array} 
        \right. $\\
        >>>FOSA: Bild Einheitssprung
\end{itemize}

\section{Ableitung an einer Sprungstelle}
Grundsignal: Einheitssprung\\
\[ \boxed{\begin{array}{ll}
\frac{d \sigma}{d t} = \dot{\sigma}(t) = 0 & \forall t \neq 0 \\
\frac{d \sigma}{d t} = \dot{\sigma}(t) = \infty & | t =   0
\end{array}
} \]
\[ \boxed{\dot{\sigma}(t) = \left\lbrace 
\begin{array}{ll}0&, t \neq 0\\\infty&, t = 0\end{array} \right.} \]
Diese Funktion heisst Diracsche Deltafunktion, Einheitsstoss, Diracstoss\\
Notation: $\delta(t)$
\[ \boxed{\dot{\sigma}(t) = \delta(t)} \]
1. Methode: \\
Linearität der Ableitung\\\\
2. Methode: \\
Graphisch\\
>>>FOSA: Bild für grafische Ableitung\\

\subsection{Fläche der Deltafunktion}
\[ \boxed{\int\limits_{-\infty}^{\infty} \delta(t) ~ dt = 1} \]
>>>FOSA: Bild Fläche

\subsection{Ausblendeeigenschaft der Deltafunktion}
\[ \boxed{\delta(t) \cdot f(t) = \delta(t) \cdot f(0)} \]

\subsection{Wichtigste Eigenschaft der Deltafunktion}
\[ \boxed{\int\limits_{-\infty}^{\infty} \delta(t) \cdot f(t) ~ dt = f(0)} \]

\section{Systeme}
\begin{figure}[h!]
    \centering
    >>>FOSA: Grafik zu Systemen
    \caption{$v(t) = \mathcal{H}\lbrace u(t) \rbrace$}
    \label{fig:system}
\end{figure}
Bsp: 
\begin{itemize}
  \item Verstärkung \\
        $v(t) = \mathcal{H} \lbrace u(t) \rbrace = c \cdot u(t)$
  \item Zeitliche Verschiebung \\
        $v(t) = \mathcal{H} \lbrace u(t) \rbrace = u(t - t_0)$
  \item Überlagerung / Superposition \\
        $v(t) = \mathcal{H} \lbrace u_1(t), u_2(t) \rbrace = u_1(t) + u_2(t)$
\end{itemize}

\subsection{Lineare zeitinvariante Systeme (LZI Systeme, LTI systems)}
\subsubsection{Linearität}
\begin{figure}[h!]
    \centering
    >>>FOSA: Grafik zu Linearität von Systemen
    \caption{}
    \label{fig:linsystem}
\end{figure}
\[ \boxed{\mathcal{H} \lbrace c \cdot u(t) \rbrace 
= c \cdot \mathcal{H} \lbrace u(t) \rbrace} \]
\begin{figure}[h!]
    \centering
    >>>FOSA: Grafik zu Linearität von Systemen2
%     \caption{}
    \label{fig:linsystem2}
\end{figure}
\[ \boxed{\mathcal{H} \lbrace u_1(t) + u_2(t) \rbrace 
= \mathcal{H} \lbrace u(t) \rbrace + \mathcal{H} \lbrace u(t) \rbrace} \]

\subsubsection{Zeitinvarianz}
\begin{figure}[h!]
    \centering
    >>>FOSA: Grafik zu Zeitinvarianz von Systemen
    \caption{}
    \label{fig:timesystem}
\end{figure}
\[ \boxed{\mathcal{H} \lbrace u(t-t_0) \rbrace 
= v(t - t_0)} \]
\\\\
Wichtige Systeme sind durch DGL gegeben, hier: lineare DGL mit konstanten 
Koeffizienten. 

\section{Laplace-Transformation}
\[ \boxed{\mathcal{L} \lbrace u(t) \rbrace 
= \int\limits_{0}^{\infty} u(t) \cdot e^{-st} ~ dt \qquad , s \in \mathbb{C} 
= U(s)} \]
\[ \boxed{\begin{array}{rcl}
\text{Zeitbereich} &  & \text{Bildbereich} \\
u(t) & \laplace & U(s)
\end{array}} \]
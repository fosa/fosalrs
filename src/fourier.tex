% coding:utf-8

%FOSALRS, a LaTeX-Code for a electrical summary of control theory
%Copyright (C) 2013, Daniel Winz, Ervin Mazlagic

%This program is free software; you can redistribute it and/or
%modify it under the terms of the GNU General Public License
%as published by the Free Software Foundation; either version 2
%of the License, or (at your option) any later version.

%This program is distributed in the hope that it will be useful,
%but WITHOUT ANY WARRANTY; without even the implied warranty of
%MERCHANTABILITY or FITNESS FOR A PARTICULAR PURPOSE.  See the
%GNU General Public License for more details.
%----------------------------------------

\chapter{Fouriertransformation}


\subsection{Fourierreihe}
Periodische Funktionen\\
Sein $u(t)$ eine $T$-periodische Funktion. D.h. 
\[ u(t + T) = u(t) \]
Die Funktion lässt sich in eine Fourier-Reihe entwickeln: 
\[ u(t) = \frac{a_0}{2} + \sum\limits_{n = 1}^{\infty} 
(a_n \cdot \cos(n \cdot \omega_0 \cdot t) 
+ b_n \cdot \sin(n \cdot \omega_0 \cdot t)) \]
Hierbei ist $\omega_0 = \frac{2 \pi}{T}$ die Grundschwingung. \\
Die Funktion lässt sich auffassen als Überlagerung von harmonischen 
Schwingungen der Grundfrequenz $\omega_0$ und ganzzahligen Vielfachen 
$n \cdot \omega_0$ der Grundfrequenz. Dies sind die Oberschwingungen oder 
Obertöne. \\
Frage: Kann msn nicht periodische Signale auffassen als Überlagerung von 
harmonischen Schwingungen? \\
Bsp. Rechteckschwingung, $T = 2 \pi, \omega_0 = 1$\\
\[ u(t) = \frac{1}{2} + \frac{2}{\pi} \cdot \sin(t) 
+ \frac{2}{3\pi} \cdot \sin(3t) 
+ \frac{2}{5\pi} \cdot \sin(5t) + \ldots \]
\[ u(t) = \frac{1}{2} + \sum\limits_{k = 0}^{\infty} 
\left(\frac{2}{\pi} \cdot \frac{\sin((2 k + 1) \cdot t)}{2 k + 1}\right) \]
D.h. 
\[ a_n = 0 \qquad \text{für alle }n = 1, 2, 3, \ldots \]
\[ a_0 = 1 \]
\[ b_n = \left\lbrace
\begin{array}{l}
\frac{2}{n \cdot \pi}, \qquad n \text{ ungerade}
0, \qquad n \text{ gerade}
\end{array}
\right. \]
\subsubsection{Formeln zur Berechnung der Koeffizienten}
\[ a_0 = \frac{2}{T} \int\limits_{-\frac{T}{2}}^{\frac{T}{2}} u(t) ~ dt \]
\[ a_0 = \frac{2}{T} \int\limits_{-\frac{T}{2}}^{\frac{T}{2}} u(t) 
\cos(n \cdot \omega_0 \cdot t) ~ dt \]
\[ b_0 = \frac{2}{T} \int\limits_{-\frac{T}{2}}^{\frac{T}{2}} u(t) 
\sin(n \cdot \omega_0 \cdot t) ~ dt \]

\subsection{Komplexe oder Exponentialform der Fourier-Reihe}
Erinnerung: Eulersche Formel: 
\[ e^{j x} = \cos(x) + j \sin(x) \qquad , x \in \mathbb{R} \]
\[ e^{- j x} = \cos(-x) + j \sin(-x) \]
\[ e^{- j x} = \cos(x) - j \sin(x) \]
\[ e^{j x} = \cos(x) + j \sin(x) \]
\[ e^{jx} + e^{-jx} = 2 \cos(x) \]
\[ \boxed{\cos(x) = \frac{1}{2} (e^{jx} + e^{-jx})} \]
\[ e^{jx} - e^{-jx} = 2 j \sin(x) \]
\[ \boxed{\sin(x) = \frac{1}{2j} (e^{jx} - e^{-jx})} \]

\subsubsection{Definition}
 Die Fourier-Transfoermierte des Signals $u(t)$ ist 
\[ \boxed{U(\omega) = \mathcal{F}\{ u(t) \} = \int\limits_{-\infty}^{\infty} 
u(t) e^{-\omega t} ~dt } \]
Man nenn $U(\omega)$ Spektraldichte oder Spektralfunktion. 

\subsubsection{Umkehrformel}
\[ \boxed{u(t) = \frac{1}{2 \pi} \int\limits_{-\infty}^{\infty} 
U(\omega) e^{\omega t j} ~ d\omega} \]

\subsection{Definition für diskrete Spektren}


\subsection{Definition für kontinuierliche Spektren}


\subsection{Exponentialform (komplexe Form)}


\subsection{Inverse Fouriertransformation}
Siehe HErleitung der Fouriertransformation: 
\[ u(t) = \frac{1}{2 \pi}  \int\limits_{-\infty}^{\infty} U(\omega) \cdot e^{\omega t j} ~ d\omega \]
\[ U(\omega) = \mathcal{F} \{ u(t) \} \]
\[ u(t) = \mathcal{F}^{-1} \{ U(\omega) \} \]

\subsection{Symmetrie}
\[ U(\omega) = \int\limits_{-\infty}^{\infty} u(t) \cdot e^{-\omega t j} ~ dt \]
Variabeln umbenennnen: 
\[ t \to -\omega \qquad \omega \to t \]
\[ u(-\omega) = \frac{1}{2 \pi} \underbrace{\int\limits_{-\infty}^{\infty} 
U(t) \cdot e^{t (-\omega) j} ~ dt}_{\mathcal{F} \{ U(t) \}} \]
\[ \mathcal{F} \{ U(t) \} = 2 \pi \cdot u(-\omega) \]
Also: 
\[ \boxed{\begin{array}{lll}
u(t) & \laplace & U(\omega)\\
U(t) & \laplace & 2 \pi \cdot u(-\omega)
\end{array}} \]


\subsection{Transformationsregeln}

\subsubsection{Zeitverschiebung}
\[ u(t - t_0) ~ \laplace ~ e^{-t_0 \omega j} \cdot U(\omega) \]
\subsubsection{Differentiationssatz}
\[ \dot{u}(t) ~ \laplace ~ \omega j \cdot U(\omega) \]
Mehrmalige Anwendung
\[ \ddot{u}(t) ~ \laplace ~ (\omega j)^2 \cdot U(\omega) = -\omega^2 U(\omega) \]
\[ u^{(n)}(t) ~ \laplace ~ (\omega j)^n \cdot U(\omega) \]


\subsubsection{Integrationssatz}


\subsubsection{Faltungsatz}
Faltung: 
\[u(t) * v(t) = \int\limits_{-\infty}^{\infty} u(\tau) \cdot v(\tau) ~ d \tau  \]
Zeitbereich: 
\[ u(t) * v(t) ~ \laplace ~ U(\omega) \cdot V(\omega) \]
Frequenzbereich:
\[ u(t) \cdot v(t) ~ \laplace ~ \frac{1}{2 \pi} U(\omega) * V(\omega) \]

\subsection{Frequenzgang}
Definition: $G(\omega j)$ heisst der Frequenzgang des Systems. 


\section{Partialbruchzerlegung}
Die Partialbruchzerlegung erlaubt es komplizierte Übertragungsfunktionen
so zu zerlegen, dass einfache Übertragungsglider daraus erkannt werden 
können. Dies ermöglicht es, dass weitere Berechnungen auf die einzelnen
Glieder angewandet werden können, etwa für die Rücktransformation in 
den Zeitbereich. Um eine solche Partialbruchzerlegung durchzuführen 
bedarf es der folgenden Schritte
%
\begin{enumerate}
    \item Polstellen finden (Linearfaktorzerlegung)
    \item Summenzerlegung
    \item Vereinfachung
    \item Gleichungssystem aufstellen
    \item Koeffizientenverleich durchführen
\end{enumerate}
%
Die Partialbruchzerlegung kann aber auch direkt mit dem Taschenrechner
erfolgen in nur einem Schritt mit dem Befehl \verb?expand()?.

\subsubsection{Polstellen finden}
In einem ersten Schritt zerlegt man das Nennerpolynom der vorliegenden 
Übertragungsfunktion mittels der Linearfaktorzerlegung.
\[  
    U(s) 
        = \frac{Z(s)}{N(s)} 
        \Rightarrow N(s) 
        = (s+a_0)(s+a_1)(s+a_2) \dots	
\]
Jeder dieser Faktoren zeigt eine Polstelle der Übertragungsfunktion auf.

\subsubsection{Summenzerlegung}
Die Faktoren, welche man per Linearfaktorzerlegung erhalten hat, werden
nun getrennt und zu separaten Nennern einer neuen Summe, welche die
ursprüngliche Übertragungsfunktion ersetzt. Die Zähler dieser neuen
Summanden bilden die Unbekannten, welche es zu ermitteln gilt.
\[  
    U(s) 
        = \frac{Z(s)}{N(s)} 
        = \frac{A}{(s+a_0)} + \frac{B}{(s+a_1)} + \frac{C}{(s+a_2)} + \dots 
\]
Sollten identische Faktoren zustandekommen und eine quadratische 
Funktion $(s+a_n)^2$ ergeben, so können diese Faktoren in einem Nenner
zusammengefasst werden. Hierbei gilt es zu beachten, dass
der Ansatz für den Zähler dann keine Konstante mehr darstellt, sondern 
eine lineare Funktion der Art $(As + B)$. Dies lässt sich beliebig erweitern
für höhere Grade des Nennerpolynoms.
\[  
    \frac{A}{(s+a_0)} 
    \rightarrow \frac{(As + B)}{(s+a)^2}
    \rightarrow \frac{(As^2 + Bs + C)}{(s+a)^3}
    \rightarrow \dots
\]

\subsubsection{Vereinfachen der Übertragungsfunktion}
Die nun vorliegende Übertragungsfunktion kann nun vereinfacht werden, indem
die Gleichung mit dem ursprünglichen Nenner erweitert wird. Auf diese
Weise kürzen sich bei allen Summanden die Nenner weg und es bleibt eine
einfache Summe ohne Brüche übrig.
\[  
    N(s) \cdot \left( 
        \frac{A}{(s+a)} 
        + \frac{B}{(s+a_1)} 
        + \frac{C}{(s+a_2)} 
    \right) 
\]
%
\[
    = A(s+a_1)(s+a_2) + B(s+a_0)(s+a_2) + C(s+a_0)(s+a_1)
\]

\subsubsection{Gleichungssystem aufstellen}
Die nun vorliegende Summe wird so umgeformt, dass jedem Grad von $s$ ein
Koeffizient zugeordnet werden kann. Die so erhaltenen Terme für den
jeweiligen Grad werden mit dem ursprünglichen Zählerpolynom der 
Übertragungsfunktion verglichen und als Gleichungssystem aufgestellt.
\[ 
    \begin{array}{r c l}  
        s^0(m_0) & = & s^0(r_1 A + r_2 B + r_3 C) \\
        s^1(m_1) & = & s^1(r_1 A + r_2 B + r_3 C) \\
        s^2(m_2) & = & s^2(r_1 A + r_2 B + r_3 C) \\
        \dots   & = & \dots
    \end{array}
\]


\subsubsection{Koeffizientenvergleich}
Das nun vorliegende Gleichungssystem kann für die Unbekannten $A,B,C$ 
aufgelöst werden. Das beidseitige Kürzen der Gleichungen mit dem 
passenden Faktor $s^n$ ergibt direkt das zu lösende Gleichungssystem. 
\[  
    \left\lvert \begin{array}{r c l}
        (m_0) & = & (r_1 A + r_2 B + r_3 C) \\
        (m_1) & = & (r_1 A + r_2 B + r_3 C) \\
        (m_2) & = & (r_1 A + r_2 B + r_3 C) \\
        \dots   & = & \dots 
    \end{array} \right\rvert
\]
Die so erhaltenen Ergebnisse können eingesetzt werden in die 
summenzerlegte Übertragungsfunktion.

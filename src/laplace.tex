% coding:utf-8

%FOSALRS, a LaTeX-Code for a electrical summary of control theory
%Copyright (C) 2013, Daniel Winz, Ervin Mazlagic

%This program is free software; you can redistribute it and/or
%modify it under the terms of the GNU General Public License
%as published by the Free Software Foundation; either version 2
%of the License, or (at your option) any later version.

%This program is distributed in the hope that it will be useful,
%but WITHOUT ANY WARRANTY; without even the implied warranty of
%MERCHANTABILITY or FITNESS FOR A PARTICULAR PURPOSE.  See the
%GNU General Public License for more details.
%----------------------------------------

\chapter{Laplacetransformation}


\subsection{Definition}
\[ 
    \mathcal{L} \lbrace u(t) \rbrace = U(s)
\]
\[ 
    U(s) = \int\limits_{0}^{\infty} u(t) \cdot e^{-st} ~ dt 
\]
\[ 
    \mathcal{L} \lbrace u(t) \rbrace 
    = \int\limits_{0}^{\infty} u(t) \cdot e^{-st} ~ dt \qquad , s \in \mathbb{C} 
    = U(s) 
\]
\[ 
    \begin{array}{rcl}
        \text{Zeitbereich} &  & \text{Bildbereich} \\
        u(t) & \laplace & U(s)
    \end{array} 
\]

\subsection{Eigenschaften }


\subsubsection{Linearität}
\[ 
    \mathcal{L}\lbrace c \cdot u(t) \rbrace 
    = c \cdot \mathcal{L}\lbrace u(t) \rbrace 
\]
\[ 
    \mathcal{L}\lbrace u_1(t) \cdot u_2(t) \rbrace 
    = \mathcal{L}\lbrace u_1(t) \rbrace + \mathcal{L}\lbrace u_2(t) \rbrace 
\]

\subsubsection{Differentiationssatz}
\[ 
    \mathcal{L}\lbrace \dot{u}(t) \rbrace = s \cdot U(s) - u(0^-) 
\]
\[ 
    \dot{u}(t) ~ \laplace ~ s \cdot U(s) - u(0^-) 
\]
\[ 
    \dddot{u}(t) ~ \laplace ~ s^3 \cdot U(s) - s^2 \cdot u(0^-) 
    - s \cdot \dot{u}(0^-) - \ddot{u}(0^-) 
\]
Häufig: Anfangsbedingungen $= 0$.
Dann:
\[ 
    \frac{d^n}{dt^n} u(t) ~ \laplace ~ s^n \cdot U(S) 
\]

\subsubsection{Integrationssatz}
\[ 
    \int\limits_{0^-}^{t} u(t) ~ dt ~ \laplace ~ \frac{1}{s} \cdot U(s) 
\]

\subsubsection{Zeitverschiebungssatz}
Ein Signal u(t) heisst kausal, wenn $u(t) = 0$ für $t < 0$.
\[ 
    u(t - t_0) \cdot \sigma(t - t_0) ~ \laplace ~ e^{-t_0s} \cdot U(s) 
\]

\subsection{Tabelle der wichtigsten Grundfunktionen}
\[ 
    \begin{array}{c@{ ~ \laplace ~ }c}
        u(t) 
            & U(s) \\
        \delta(t) 
            & 1 \\
        \sigma(t) 
            & \frac{1}{s} \\
        t \cdot \sigma(t) 
            & \frac{1}{s^2} \\
        t^2 \cdot \sigma(t) 
            & \frac{2}{s^3} \\
        \vdots 
            & \vdots \\
        e^{a \cdot t} 
            & \frac{1}{s - a} \\
        \sin(t) 
            & \frac{1}{s^2 + 1} \\
        \cos(t) 
            & \frac{s}{s^2 + 1} \\
    \end{array} 
\]

\subsection{Inverse Laplace-Transformation}
\[ 
    \mathcal{L} \lbrace u(t) \rbrace = U(s)
\]
\[ 
    u(t) = \mathcal{L}^{-1} \lbrace U(s) \rbrace
\]

\subsubsection{Eindeutigkeitssatz}
\[ 
    U(s) = V(s) 
\]
\[ 
    u(t) = v(t) \forall t \geq 0 
\]
Notation:
\[ 
    \mathcal{L} \{ u(t) \} = U(s) 
\]
\[ 
    u(t) = \mathcal{L}^{-1} \{ U(s) \} 
\]

\subsubsection{Dämpfungssatz}
\[ 
    e^{at} \cdot u(t) ~ \laplace ~ U(s - a) 
\]
Vorsicht! Vorzeichen anders als beim Zeitverschiebungssatz.

\subsubsection{Frequenzverschiebungssatz}

\subsection{Partialbruchzerlegung}

\subsection{Anfangswertsatz}
Sei $u(t)$ an der Stelle $t = 0$ sprungstetig. Dann gilt
\[ 
    u(0^+) = \lim\limits_{s \to \infty} s \cdot U(s) 
\]
Bemerkung: \\
Wenn $U(s) = \dfrac{Z(s)}{N(s)}$ eine rationale Funktion mit
$\text{Grad}(Z(s)) < \text{Grad}(N(s)$, dann ist u(t) sprungstetig

\subsection{Endwertsatz }
Sein $u(t)$ ein Signal, für das der Endwert
$u(\infty) = \lim\limits_{t \to \infty} u(t)$ existiert. \\
Dann gilt:
\[ 
u(\infty) = \lim\limits_{s \to 0} s \cdot U(s) 
\]
Kriterium: \\
Wenn alle Polstellen von $s \cdot U(s)$ links von der imaginären Achse liegen,
dann existiert der Endwert $u(\infty)$. \\
Mit anderen Worten: Der Realteil aller Nullstellen muss negativ sein.

\subsection{DGL lösen mittels Laplace-Transformation}

\subsection{Übertragungsfunktion, Gewichtsfunktion}

\subsection{Faltung}
Definition: Die Faltung zweier Signale $u(t)$ und $v(t)$ ist
\[ 
    (u * v)(t) = \int\limits_{-\infty}^{\infty} u(\tau) v(t - \tau) ~ d \tau 
\]
Bemerkung: Sind die Signale $u(t)$ und $v(t)$ sprungstetig und kausal, dann
gilt:
\[ 
    (u * v)(t) = \int\limits_{0}^{t} u(\tau) v(t - \tau) ~ d \tau 
\]
Faltungssatz:
Sind $u(t), v(t)$ kausal, so gilt:
\[ 
    (u * v) (t) ~ \laplace ~ U(s) \cdot V(s) 
\]

\subsection{Impulsantwort}
Ist $g(t)$ die Gewichtsfunktion eines LZI Systems, so gilt
\[ 
    v(t) = (g * u)(t) 
\]
Fazit: Impulsantwort = Gewichtsfunktion

\subsection{Sprungantwort}
Definition: Sprungantwort = Antwort auf $\sigma(t) = h(t)$
(AW nicht unbedingt 0) \\
Bemerkung: Bei LZI Systemen:
\[ 
    h(t) = \int\limits_{0}^{t} g(\tau) ~d \tau 
\]
(folgt aus Integrationssatz)

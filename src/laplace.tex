% coding:utf-8

%FOSALRS, a LaTeX-Code for a electrical summary of control theory
%Copyright (C) 2013, Daniel Winz, Ervin Mazlagic

%This program is free software; you can redistribute it and/or
%modify it under the terms of the GNU General Public License
%as published by the Free Software Foundation; either version 2
%of the License, or (at your option) any later version.

%This program is distributed in the hope that it will be useful,
%but WITHOUT ANY WARRANTY; without even the implied warranty of
%MERCHANTABILITY or FITNESS FOR A PARTICULAR PURPOSE.  See the
%GNU General Public License for more details.
%----------------------------------------

\chapter{Laplacetransformation}


\section{Definition}
\[ 
    \mathcal{L} \lbrace u(t) \rbrace = U(s)
\]
\[ 
    U(s) = \int\limits_{0^-}^{\infty} u(t) \cdot e^{-st} ~ dt 
\]
\[ 
    \mathcal{L} \lbrace u(t) \rbrace 
    = \int\limits_{0^-}^{\infty} u(t) \cdot e^{-st} ~ dt \qquad , s \in \mathbb{C} 
    = U(s) 
\]
\[ 
    \begin{array}{rcl}
        \text{Zeitbereich} &  & \text{Bildbereich} \\
        u(t) & \laplace & U(s)
    \end{array} 
\]

\section{Eigenschaften }


\subsection{Linearität}
\[ 
    \mathcal{L}\lbrace c \cdot u(t) \rbrace 
    = c \cdot \mathcal{L}\lbrace u(t) \rbrace 
\]
\[ 
    \mathcal{L}\lbrace u_1(t) + u_2(t) \rbrace 
    = \mathcal{L}\lbrace u_1(t) \rbrace + \mathcal{L}\lbrace u_2(t) \rbrace 
\]

\subsection{Differentiationssatz}
\[ 
    \mathcal{L}\lbrace \dot{u}(t) \rbrace = s \cdot U(s) - u(0^-) 
\]
\[ 
    \dot{u}(t) ~ \laplace ~ s \cdot U(s) - u(0^-) 
\]
\[ 
    \ddot{u}(t) ~ \laplace ~ s^2 \cdot U(s) - s \cdot u(0^-) 
    - \dot{u}(0^-)
\]
\[ 
    \dddot{u}(t) ~ \laplace ~ s^3 \cdot U(s) - s^2 \cdot u(0^-) 
    - s \cdot \dot{u}(0^-) - \ddot{u}(0^-) 
\]
Häufig: Anfangsbedingungen $= 0$.
Dann:
\[ 
    \frac{d^n}{dt^n} u(t) ~ \laplace ~ s^n \cdot U(s) 
\]

\subsection{Integrationssatz}
\[ 
    \int\limits_{0^-}^{t} u(t) ~ dt ~ \laplace ~ \frac{1}{s} \cdot U(s) 
\]

\subsection{Zeitverschiebungssatz}
Ein Signal u(t) heisst kausal, wenn $u(t) = 0$ für $t < 0$.
\[ 
    u(t - t_0) \cdot \sigma(t - t_0) ~ \laplace ~ e^{-t_0s} \cdot U(s) 
\]

\section{Tabelle der wichtigsten Grundfunktionen}
\[ 
    \begin{array}{c@{ ~ \laplace ~ }c}
        u(t) 
            & U(s) \\
        \delta(t) 
            & 1 \\
        \sigma(t) 
            & \frac{1}{s} \\
        t \cdot \sigma(t) 
            & \frac{1}{s^2} \\
        t^2 \cdot \sigma(t) 
            & \frac{2}{s^3} \\
        \vdots 
            & \vdots \\
        t^n \cdot \sigma(t) 
            & \frac{n!}{s^{n+1}} \\
        e^{a \cdot t} 
            & \frac{1}{s - a} \\
        \sin(t) 
            & \frac{1}{s^2 + 1} \\
        \cos(t) 
            & \frac{s}{s^2 + 1} \\
    \end{array} 
\]

\section{Inverse Laplace-Transformation}
\[ 
    \mathcal{L} \lbrace u(t) \rbrace = U(s)
\]
\[ 
    u(t) = \mathcal{L}^{-1} \lbrace U(s) \rbrace
    = \frac{1}{2\pi j} \int\limits_{e-j\infty}^{e+j\infty} F(s) e^{st} ~ ds
\]

\subsection{Eindeutigkeitssatz}
\[ 
    U(s) = V(s) 
\]
\[ 
    u(t) = v(t) \qquad \forall t \geq 0 
\]

\subsection{Dämpfungssatz}
\[ 
    e^{at} \cdot u(t) ~ \laplace ~ U(s - a) 
\]
Vorsicht! Vorzeichen anders als beim Zeitverschiebungssatz.

\subsection{Frequenzverschiebungssatz}

%\section{Partialbruchzerlegung}
\section{Partialbruchzerlegung}
Die Partialbruchzerlegung erlaubt es komplizierte Übertragungsfunktionen
so zu zerlegen, dass einfache Übertragungsglider daraus erkannt werden 
können. Dies ermöglicht es, dass weitere Berechnungen auf die einzelnen
Glieder angewandet werden können, etwa für die Rücktransformation in 
den Zeitbereich. Um eine solche Partialbruchzerlegung durchzuführen 
bedarf es der folgenden Schritte
%
\begin{enumerate}
    \item Polstellen finden (Linearfaktorzerlegung)
    \item Summenzerlegung
    \item Vereinfachung
    \item Gleichungssystem aufstellen
    \item Koeffizientenverleich durchführen
\end{enumerate}
%
Die Partialbruchzerlegung kann aber auch direkt mit dem Taschenrechner
erfolgen in nur einem Schritt mit dem Befehl \verb?expand()?.

\subsubsection{Polstellen finden}
In einem ersten Schritt zerlegt man das Nennerpolynom der vorliegenden 
Übertragungsfunktion mittels der Linearfaktorzerlegung.
\[  
    U(s) 
        = \frac{Z(s)}{N(s)} 
        \Rightarrow N(s) 
        = (s+a_0)(s+a_1)(s+a_2) \dots	
\]
Jeder dieser Faktoren zeigt eine Polstelle der Übertragungsfunktion auf.

\subsubsection{Summenzerlegung}
Die Faktoren, welche man per Linearfaktorzerlegung erhalten hat, werden
nun getrennt und zu separaten Nennern einer neuen Summe, welche die
ursprüngliche Übertragungsfunktion ersetzt. Die Zähler dieser neuen
Summanden bilden die Unbekannten, welche es zu ermitteln gilt.
\[  
    U(s) 
        = \frac{Z(s)}{N(s)} 
        = \frac{A}{(s+a_0)} + \frac{B}{(s+a_1)} + \frac{C}{(s+a_2)} + \dots 
\]
Sollten identische Faktoren zustandekommen und eine quadratische 
Funktion $(s+a_n)^2$ ergeben, so können diese Faktoren in einem Nenner
zusammengefasst werden. Hierbei gilt es zu beachten, dass
der Ansatz für den Zähler dann keine Konstante mehr darstellt, sondern 
eine lineare Funktion der Art $(As + B)$. Dies lässt sich beliebig erweitern
für höhere Grade des Nennerpolynoms.
\[  
    \frac{A}{(s+a_0)} 
    \rightarrow \frac{(As + B)}{(s+a)^2}
    \rightarrow \frac{(As^2 + Bs + C)}{(s+a)^3}
    \rightarrow \dots
\]

\subsubsection{Vereinfachen der Übertragungsfunktion}
Die nun vorliegende Übertragungsfunktion kann nun vereinfacht werden, indem
die Gleichung mit dem ursprünglichen Nenner erweitert wird. Auf diese
Weise kürzen sich bei allen Summanden die Nenner weg und es bleibt eine
einfache Summe ohne Brüche übrig.
\[  
    N(s) \cdot \left( 
        \frac{A}{(s+a)} 
        + \frac{B}{(s+a_1)} 
        + \frac{C}{(s+a_2)} 
    \right) 
\]
%
\[
    = A(s+a_1)(s+a_2) + B(s+a_0)(s+a_2) + C(s+a_0)(s+a_1)
\]

\subsubsection{Gleichungssystem aufstellen}
Die nun vorliegende Summe wird so umgeformt, dass jedem Grad von $s$ ein
Koeffizient zugeordnet werden kann. Die so erhaltenen Terme für den
jeweiligen Grad werden mit dem ursprünglichen Zählerpolynom der 
Übertragungsfunktion verglichen und als Gleichungssystem aufgestellt.
\[ 
    \begin{array}{r c l}  
        s^0(m_0) & = & s^0(r_1 A + r_2 B + r_3 C) \\
        s^1(m_1) & = & s^1(r_1 A + r_2 B + r_3 C) \\
        s^2(m_2) & = & s^2(r_1 A + r_2 B + r_3 C) \\
        \dots   & = & \dots
    \end{array}
\]


\subsubsection{Koeffizientenvergleich}
Das nun vorliegende Gleichungssystem kann für die Unbekannten $A,B,C$ 
aufgelöst werden. Das beidseitige Kürzen der Gleichungen mit dem 
passenden Faktor $s^n$ ergibt direkt das zu lösende Gleichungssystem. 
\[  
    \left\lvert \begin{array}{r c l}
        (m_0) & = & (r_1 A + r_2 B + r_3 C) \\
        (m_1) & = & (r_1 A + r_2 B + r_3 C) \\
        (m_2) & = & (r_1 A + r_2 B + r_3 C) \\
        \dots   & = & \dots 
    \end{array} \right\rvert
\]
Die so erhaltenen Ergebnisse können eingesetzt werden in die 
summenzerlegte Übertragungsfunktion.


\section{Anfangswertsatz}
Sei $u(t)$ an der Stelle $t = 0$ sprungstetig. Dann gilt
\[ 
    u(0^+) = \lim\limits_{s \to \infty} s \cdot U(s) 
\]
Bemerkung: \\
Wenn $U(s) = \dfrac{Z(s)}{N(s)}$ eine rationale Funktion  und der Zählergrad 
kleiner als der Nennergrad ist, dann ist u(t) sprungstetig

\section{Endwertsatz}
Sei $u(t)$ ein Signal, für das der Endwert
$u(\infty) = \lim\limits_{t \to \infty} u(t)$ existiert. \\
Dann gilt:
\[ 
u(\infty) = \lim\limits_{s \to 0} s \cdot U(s) 
\]
Kriterium: \\
Wenn alle Polstellen von $s \cdot U(s)$ links von der imaginären Achse liegen,
dann existiert der Endwert $u(\infty)$. \\
Mit anderen Worten: Der Realteil aller Pole muss negativ sein.

\section{DGL lösen mittels Laplace-Transformation}
\begin{enumerate}
    \item Laplace Transformation anwenden
    \item Auflösen nach $V(s)$
    \item Eingangssignal transformieren und einsetzen
    \item Rücktransformation
\end{enumerate}

\section{Übertragungsfunktion, Gewichtsfunktion}
\[ 
    V(s) = \underbrace{\frac{1}{RC \cdot s + 1}}_{G(s)} \cdot U(s) 
\]
$G(s)$: Übertragungsfunktion\\
Die Zeitfunktion $g(t)$ von $G(s)$ heisst Gewichtsfunktion\\

\section{Faltung}
Definition: Die Faltung zweier Signale $u(t)$ und $v(t)$ ist
\[ 
    (u * v)(t) = \int\limits_{-\infty}^{\infty} u(\tau) v(t - \tau) ~ d \tau 
\]
Bemerkung: Sind die Signale $u(t)$ und $v(t)$ sprungstetig und kausal, dann
gilt:
\[ 
    (u * v)(t) = \int\limits_{0}^{t} u(\tau) v(t - \tau) ~ d \tau 
\]
Faltungssatz:
Sind $u(t), v(t)$ sprungstetig und kausal, so gilt:
\[ 
    (u * v) (t) ~ \laplace ~ U(s) \cdot V(s) 
\]

\section{Impulsantwort}
Ist $g(t)$ die Gewichtsfunktion eines LZI Systems, so gilt
\[ 
    y(t) = (g * u)(t) 
\]
Fazit: Impulsantwort = Gewichtsfunktion

\section{Sprungantwort}
Definition: Sprungantwort $h(t)$ = Antwort auf $\sigma(t)$
(AW nicht unbedingt 0) \\
Bemerkung: Bei LZI Systemen:
\[ 
    h(t) = \int\limits_{0}^{t} g(\tau) ~d \tau 
\]
(folgt aus Integrationssatz)

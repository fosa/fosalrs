\section{Bode-Diagramm}
Das Bode-Diagramm ist eine kombinierte Darstellung von Amplituden- und
Phasengang, welches zudem logarithmisch skaliert ist. Die logarithmische
Skalierung erlaubt es Verläufe von Teilsystemen zu addieren statt zu 
multiplizieren. Dies rührt her von der Regel
\[
    \log_a(x \cdot y) = \log_a(x) + \log_a(y)  
\]
Das Bode-Diagramm zeigt sowohl den Amplituden- als auch
den Phasengang in Funktion der Kreisfrequenz $\omega$.
%
\begin{figure}[h!]
    \centering
    \begin{tikzpicture}
        % Coordinaten
        \coordinate (a1) at (0,0);
        \coordinate (a2) at (0,-3);
        % Amplitudengang
        \begin{scope}[shift={(a1)}]
            % Koordinatensystem 
            \draw[->] (-0.1,0) -- (6.5,0) node[below] {$\omega$};
            \draw[->] (0,-2.0) -- (0,2) node[left] 
                {$20 \log \left( \left\lvert G \right\rvert \right)$};
            % x-Tics
            \draw[] (1,0.1) -- (1,-0.1);
            \draw[] (2,0.1) -- (2,-0.1);
            \draw[] (3,0.1) -- (3,-0.1);
            \draw[] (4,0.1) -- (4,-0.1);
            \draw[] (5,0.1) -- (5,-0.1);
            % y-Tics
            \draw[] (0.1,1.5) -- (-0.1,1.5) node[left] {$60$ dB};
            \draw[] (0.1,1.0) -- (-0.1,1.0) node[left] {$40$ dB}; 
            \draw[] (0.1,0.5) -- (-0.1,0.5) node[left] {$20$ dB};
            \draw[] (0.1,0.0) -- (-0.1,0.0) node[left] {$0$ dB};
            \draw[] (0.1,-0.5) -- (-0.1,-0.5) node[left] {$-20$ dB};;
            \draw[] (0.1,-1.0) -- (-0.1,-1.0) node[left] {$-40$ dB};;
            \draw[] (0.1,-1.5) -- (-0.1,-1.5) node[left] {$-60$ dB};;
            % Amplitudengang
            \draw[blue, thick] (0,0) -- (1,0) -- (4,-1.5) 
                node[below, midway] {$G_1$};
            \draw[green, thick] (0,1.5) -- (2,1.5) -- (6,-0.5)
                node[above, midway] {$G_2$};
            \draw[red, thick] (0,1.5) -- (1,1.5) -- (2,1) -- (4.5,-1.5)
                node[above right] {$G_1+G_2$};
        \end{scope}
        % Phasengang
        \begin{scope}[shift={(a2)}]
            % Koordinatensystem 
            \draw[->] (-0.1,0) -- (6.5,0) node[below] {$\omega$};
            \draw[->] (0,-3) -- (0,0.5) node[left] {$\angle\varphi$};
            % x-Tics
            \draw[] (1,0.1) -- (1,-0.1);
            \draw[] (2,0.1) -- (2,-0.1);
            \draw[] (3,0.1) -- (3,-0.1);
            \draw[] (4,0.1) -- (4,-0.1);
            \draw[] (5,0.1) -- (5,-0.1);
            % y-Tics
            \draw[] (0.1,0) -- (-0.1,0) node[left] {$0^\circ$};
            \draw[] (0.1,-0.5) -- (-0.1,-0.5) node[left] {$-45^\circ$};
            \draw[] (0.1,-1.0) -- (-0.1,-1.0) node[left] {$-90^\circ$};
            \draw[] (0.1,-1.5) -- (-0.1,-1.5) node[left] {$-135^\circ$};
            \draw[] (0.1,-2.0) -- (-0.1,-2.0) node[left] {$-180^\circ$};
            \draw[] (0.1,-2.5) -- (-0.1,-2.5) node[left] {$-225^\circ$};
            % Phasengang
            \draw[blue, thick] (0,0) -- (2,-1) 
                node[above] {$G_1$} -- (6,-1);
            \draw[green, thick] (0,0) -- (1,0) -- (3,-1) 
                node[above] {$G_2$} -- (6,-1);
            \draw[red, thick] (0,0) -- (1,-0.5) -- (2,-1.5) -- (3,-2) 
                -- (6,-2) node[above] {$G_1+G_2$};
        \end{scope}
    \end{tikzpicture}
\end{figure}

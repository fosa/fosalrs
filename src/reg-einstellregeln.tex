% coding:utf-8

%FOSALRS, a LaTeX-Code for a electrical summary of control theory
%Copyright (C) 2013, Daniel Winz, Ervin Mazlagic

%This program is free software; you can redistribute it and/or
%modify it under the terms of the GNU General Public License
%as published by the Free Software Foundation; either version 2
%of the License, or (at your option) any later version.

%This program is distributed in the hope that it will be useful,
%but WITHOUT ANY WARRANTY; without even the implied warranty of
%MERCHANTABILITY or FITNESS FOR A PARTICULAR PURPOSE.  See the
%GNU General Public License for more details.
%----------------------------------------

\section{Einstellregeln PID Regler}

\subsection{Ziegler-Nichols}
Die Regelstrecke wird als $PT_1$ Glied mit Totzeit betrachtet. 
\[ G_S(s) = \frac{K_S \cdot e^{-T_t}}{1 + s \cdot T_S} \]

\subsubsection{Übergangsmethode}
\begin{table}[h!]
    \[
        \begin{array}{lccc}
            &
                K_R &
                T_N &
                T_V \\\\
            \text{P} &
                \frac{T_S}{K_S \cdot T_t} &
                - &
                - \\\\
            \text{PI} &
                \frac{0.9 \cdot T_S}{K_S \cdot T_t} &
                3.3 \cdot T_t &
                - \\\\
            \text{PID} &
                \frac{1.2 \cdot T_S}{K_S \cdot T_t} &
                2.0 \cdot T_t &
                0.5 \cdot T_t \\\\
        \end{array}
    \]
\end{table}

\subsubsection{Stabilitätsgrenzenmethode}
$K_R$ wird erhöht, bis eine Schwingung auftritt. $\to K_R = K_{krit}$\\
Dabei wird die Schwingtauer $T_{krit}$ bestimmt. \\\\
Alternativ können $K_{krit}$ und $T_{krit}$ auch aus dem Bodediagramm oder 
aus der Nyquist Ortskurve gelesen werden. $T_{krit} = \frac{2 \pi}{\omega_{krit}}$
\begin{table}[h!]
    \[
        \begin{array}{lccc}
            &
                K_R &
                T_N &
                T_V \\\\
            \text{P} &
                0.5 \cdot K_{krit} &
                - &
                - \\\\
            \text{PI} &
                0.45 \cdot K_{krit} &
                0.83 \cdot T_{krit} &
                - \\\\
            \text{PID} &
                0.6 \cdot K_{krit} &
                0.5 \cdot T_{krit} &
                0.125 \cdot T_{krit} \\\\
        \end{array}
    \]
\end{table}


\subsection{Chien, Hrones und Reswick}


\subsection{Aström}



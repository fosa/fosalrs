\section{Linearisierung}
Liegt ein nicht-lineares System vor, so kann dieses für einen
bestimmten Arbeitspunkt linearisiert werden. So wird das
nicht-lineare System in ein LZI-Glied überführt.

\subsection{Gleichgewichtslage (stationärer Zustand)}
Die Gleichgewichtslage bezeichnet einen Zustand, an dem ein 
konstantes $x$ ein konstantes $y$ ergibt. Jene Werte, bei denen 
diese Gleichheit gilt, werden als $\bar{x}$ bzw. $\bar{y}$ 
notiert. Da sowohl $x=const.$ als auch $y=const.$ gilt, sind
sämtliche Ableitungen von $x$ und $y$ null.
\[  
    \bar{x}, \bar{y}
        \quad \Rightarrow \quad
        \dot{x} = 0,
        \ddot{x} = 0,    
        \dots,
        x^{(n)} = 0,
        \dot{y} = 0,
        \ddot{y} = 0,
        \dots,
        y^{(n)} = 0
\]

\subsection{Totales Differential}
Um eine Linearisierung mittels des totalen Differentials durchzuführen,
muss die Funktion hinsichtlich der fixierten veränderlichen $x$ 
aufgestellt werden. Diese wird dann gleich null gesetzt. 
Der Arbeitspunkt $\bar{x}$ wird dann an jeder Stelle eingesetzt für die 
fixierte Grösse $x$. 
\[ 
    f(\dddot{x}, \ddot{x}, \dot{x}, x) = 0 \rightarrow \bar{x}
\]
\[ 
    \begin{array}{r l} 
		\displaystyle 
            \left. \frac{\partial f}{\partial \dddot{x}}
                \right\rvert_{\bar{x}} \cdot \Delta \dddot{x}
            + \left.\frac{\partial f}{\partial \ddot{x}}
                \right\rvert_{\bar{x}} \cdot \Delta \ddot{x}
            + \left.\frac{\partial f}{\partial \dot{x}}
                \right\rvert_{\bar{x}} \cdot \Delta \dot{x}
            + \left.\frac{\partial f}{\partial x}
                \right\rvert_{\bar{x}} \cdot \Delta x 
            = 0\\
    \end{array} 
\]



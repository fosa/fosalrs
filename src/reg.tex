% coding:utf-8

%FOSALRS, a LaTeX-Code for a electrical summary of control theory
%Copyright (C) 2013, Daniel Winz, Ervin Mazlagic

%This program is free software; you can redistribute it and/or
%modify it under the terms of the GNU General Public License
%as published by the Free Software Foundation; either version 2
%of the License, or (at your option) any later version.

%This program is distributed in the hope that it will be useful,
%but WITHOUT ANY WARRANTY; without even the implied warranty of
%MERCHANTABILITY or FITNESS FOR A PARTICULAR PURPOSE.  See the
%GNU General Public License for more details.
%----------------------------------------

\section{Systemtheorie}

\subsection{Steuerung}
\begin{figure}[h!]
    \centering
    >>>FOSA: Grafik zu Steuerung
    \caption{Steuerung}
    \label{fig:control}
\end{figure}
Einfach zu Implementieren
keine Korrektur von Störungen

\subsection{Regelung}
\begin{figure}[h!]
    \centering
    >>>FOSA: Grafik zu Systemtheorie
    \caption{Regelkreis}
    \label{fig:regsystem}
\end{figure}
\begin{tabular}{@{}ll}
$r$ & Sollwert, Führungsgrösse (manchmal $w$) \\
$e$ & Regeldifferenz \\
$u$ & Stellgrösse \\
$y$ & Regelgrösse, Ausgangsgrösse \\
$d$ & Störgrösse (Eher v bei Rauschen)
\end{tabular}
\[ \boxed{e = r - y} \]
\[ \boxed{e(t) = r(t) - y(t)} \]
\begin{figure}[h!]
    \centering
    >>>FOSA: Grafik zu Regelkreis mit Feed forward
    \caption{Regelkreis mit Feed forward}
    \label{fig:feedforward}
\end{figure}
2 Degrees of freedom

\section{Eigenschaften eines Regelkreises}
Ein Regler kann nur entweder für das Führungsverhalten oder für das 
Störverhalten optimiert werden. 

\subsection{Führungsverhalten}
\begin{figure}[h!]
    \centering
    >>>FOSA: Grafik zu Führungsverhalten
    \caption{Führungsverhalten}
    \label{fig:fhr}
\end{figure}
\begin{itemize}
  \item maximales Überschwingen
  \item stationäre Genauigkeit
  \item Anregelzeit (Anstiegszeit (Reaktionszeit))
\end{itemize}

\subsection{Störverhalten}
\begin{figure}[h!]
    \centering
    >>>FOSA: Grafik zu Störverhalten
    \caption{Störverhalten}
    \label{fig:str}
\end{figure}
\begin{itemize}
  \item maximales Unterschwingen
  \item Ausregelzeit
  \item Stationäre Regelabweichung
\end{itemize}

\section{LZI -- Lineare zeitinvariante Elemente}
bounded Input $\rightarrow$ bounded output -- BIBO\\
P-Anteil: Grundsätzliche Stabilisierung \\
I-Anteil: Reduktion der Regeldifferenz auf 0 \\
D-Anteil: Becshleunigung des Reglers

\subsection{Prüfung auf Linearität}
\begin{enumerate}
  \item Überlagerungsprinzip erfüllt 
        \[ y = f(u) \qquad f(u_1 + u_2) = y_1 + y_2 \]
  \item Verstärkungsprinzip erfüllt
        \[ y = f(u) \qquad f(k \cdot u) = k \cdot y \]
\end{enumerate}

\subsection{Prüfung auf Zeitvarianz}
\[ y(t) = f(u(t) \]
\[ y(t - T) = f(u(t - T)) \]

\section{Wichtige Gleichungen für die Modellierung dynamischer Systeme}

Bilanzgleichungen, Erhaltungssätze
\subsection{Elektrisches Netzwerk}
Ladungserhaltung \\
Energieerhaltung \\
$\Rightarrow$ Kirchhoffsche Regeln \\
Knotenregel $\sum I = 0$ \\
Maschenregel $\sum U = 0$ \\\\
Widerstand R
\[ \boxed{u_R(t) = R \cdot i_R(t)} \]
\[ \boxed{i_R(t) = \frac{1}{R} \cdot u_R(t)} \]
Kapazität
\[ \boxed{u_C(t) 
= \frac{1}{C} \cdot \int\limits_{0}^{t} i_C(\tau) ~ d\tau + u_C(0)} \]
\[ \boxed{\dot{u}_C(t) = \frac{1}{C} \cdot i_C(t)} \]
\[ \boxed{i_C(t) = C \cdot \frac{d u_C(t)}{d t}} \]
Induktivität
\[ \boxed{u_L(t) = L \cdot \frac{d i}{d t}} \]
\[ \boxed{i_L(t) 
= \frac{1}{L} \cdot \int\limits_{0}^{t} u_L(\tau) ~ d\tau + i_L(0)} \]

\subsection{Mechanisch (Translation)}
Newton'sche Gesetze (skalar)
\[ \boxed{\frac{d(m \cdot v)}{dt} = \sum F_i} \]
\[ \boxed{\dot{m} \cdot v + m \cdot \dot{v} = \sum F_i} \]
m = konst: 
\[ \boxed{m \cdot \dot{v} = \sum F_i} \]
\[ \boxed{x, \qquad \dot{x} = v, \qquad \ddot{x} = \dot{v} = a} \]
\[ \boxed{m \cdot \ddot{x} = \sum F_i} \]

\subsection{Mechanisch (Rotation)}
\[ \boxed{\frac{d(I \cdot \omega}{dt} = \sum M_i} \]
\[ \boxed{I = konst. } \]
\[ \boxed{I \cdot \dot{\omega} = \sum M_i} \]
\[ \boxed{\varphi, \qquad \dot{\varphi} = \omega, 
\qquad \ddot{\varphi} = \dot{\omega} = \alpha} \]

\subsection{Thermisch}
\[ \boxed{\frac{d(c \cdot M \cdot \vartheta)}{dt} = \sum Q_i} \]

\subsection{Hydraulisches System}
inkompressibel \\
Volumen / Massenbilanz
\[ \boxed{\frac{dV}{dt} = \sum q_i} \]

\subsection{pneumatisch}
\[ \boxed{} \]
\subsection{Konzentration}
\[ \boxed{} \]

\section{Definitionen}

\subsection{Steuerung}
Eine Steuerung ist eine Vorrichtung, welche einen Eingabewert
auf bestimmte Weise umwandelt zu einem Ausgabewert.

\begin{signalflow}[node distance=15mm]{Steuerung}
    % Elemente
    \tikzgrid{
        \node[input]    (in)    {}          &
        \node[delay]    (d1)    {$G_1$}     &
        \node[delay]    (d2)    {$G_2$}     &  
        \node[output]   (out)   {}
    }
    % Pfade
    \path[r>] (in) -- node[above]{$w$} (d1);
    \path[r>] (d1) -- node[above]{$x$} (d2);
    \path[r>] (d2) -- node[above]{$y$} (out);
\end{signalflow}

\subsection{Regelung}

\begin{signalflow}[node distance=15mm]{}
    % Elemente
    \tikzgrid{
        \node[input]    (in)    {}          &
        \node[adder]    (a1)    {}          &
        \node[delay]    (d1)    {$G_1$}     &
        \node[delay]    (d2)    {$G_2$}     &  
        \node[node]     (n1)    {}          &
        \node[output]   (out)   {}
    }
    % Pfade
    \path[r>] (in) -- node[above]{$w$} (a1);
    \path[r>] (a1) -- node[above]{$e$} (d1);
    \path[r>] (d1) -- node[above]{$x$} (d2);
    \path[r>] (d2) -- node[above]{$y$} (out);
    %\coordinate (d) at ($ (n1) + (0,-0.5) $);
    \path[r>] (n1) -- ++(0,-0.5)  -| node[left, pos=0.95] {$-$} (a1);
\end{signalflow}











































\subsection{Führungsverhalten}

\subsection{Störungsverhalten}

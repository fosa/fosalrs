\section{Definitionen}

\subsection{Steuerung}
Eine Steuerung ist eine Vorrichtung, welche einen Eingabewert
auf bestimmte Weise umwandelt zu einem Ausgabewert. Das bedeutet,
dass jegliche Einflüsse am Ausgang nicht berücksichtigt werden.
%
\begin{figure}[h!]
	\centering
    \begin{signalflow}[node distance=15mm]
        % Elemente
        \tikzgrid{
            \node[input]    (in)    {}          &
            \node[delay]    (d1)    {$G_1$}     &
            \node[delay]    (d2)    {$G_2$}     &  
            \node[output]   (out)   {}
        }
        % Pfade
        \path[r>] (in) -- node[above]{$w$} (d1);
        \path[r>] (d1) -- node[above]{$x$} (d2);
        \path[r>] (d2) -- node[above]{$y$} (out);
    \end{signalflow}
\end{figure}
%
Die Übertragungsfunktion eines solchen Kreises lautet dann
$G(s) = G_1G_2$

\subsection{Regelung}
Eine Regelung ist eine Vorrichtung, welche einen Eingabewert mit
dem Ausgabewert vergleicht. Anhand der Differenz wird dann eine 
Veränderung am Ausgang vorgenommen.
%
\begin{figure}[h!]
    \begin{signalflow}[node distance=15mm]
        % Elemente
        \tikzgrid{
            \node[input]    (in)    {}          &
            \node[adder]    (a1)    {}          &
            \node[delay]    (d1)    {$G_1$}     &
            \node[delay]    (d2)    {$G_2$}     &  
            \node[node]     (n1)    {}          &
            \node[output]   (out)   {}
        }
        % Pfade
        \path[r>] (in) -- node[above]{$w$} (a1);
        \path[r>] (a1) -- node[above]{$e$} (d1);
        \path[r>] (d1) -- node[above]{$x$} (d2);
        \path[r>] (d2) -- node[above]{$y$} (out);
        \path[r>] (n1) -- ++(0,-0.5)  -| node[left, pos=0.95] {$-$} (a1);
    \end{signalflow}
\end{figure}
%
Die Übertragungsfunktion eines solchen Kreises lautet
$G(s)=\frac{G_1G_2}{1+G_2G_2}$

\subsection{Grössen eines Regelkreises}
\begin{figure}[h!]
    \begin{signalflow}[node distance=15mm]{}
        % Elemente
        \tikzgrid{
            \node[input]    (in)    {}          &
            \node[adder]    (a1)    {}          &
            \node[delay]    (d1)    {Regler}    & &
            \node[delay]    (d2)    {Strecke}   &  
            \node[node]     (n1)    {}          &
            \node[output]   (out)   {}
        }
        % Pfade
        \path[r>] (in) -- node[above]{$w$} (a1);
        \path[r>] (a1) -- node[above]{$e$} (d1);
        \path[r>] (d1) -- node[above]{$x$} (d2);
        \path[r>] (d2) -- node[above]{$y$} (out);
        \path[r>] (n1) -- ++(0,-0.5)  -| node[left, pos=0.95] {$-$} (a1);
        \path[r>] (d2) ++(0,1) -- node[right]{$z$} (d2); 
    \end{signalflow}
\end{figure}
%
\begin{table}[h!]
    \centering
    \begin{tabular}{c l l}
        Variable    & Bezeichnung       & Beschreibung \\
        \hline
        $w$         & Führungsgrösse    & Was zu erreichen ist (SOLL) \\
        $e$         & Regelfehler       & Differenz von SOLL und IST \\
        $x$         & Stellgrösse       & erfolderliche Steuergrösse \\
        $y$         & Regelgrösse       & Was aktuell gilt (IST) \\
        $z$         & Störgrösse        & Störungen \\
    \end{tabular}
\end{table}

\subsection{Führungsverhalten}
Als Führungsverhalten bezeichnet man das Verhalten des geschlossenen
Regelkreises, welches sich ergibt, wenn ausschliesslich die 
Führungsgrösse auf den Regelkreis einwirkt. Typisch wird hierfür ein 
Einheitssprung eingespielt. Alle anderen Eingaben, wie etwa die 
Störgrösse, sind dabei Null. 

\begin{figure}[h!]
    \begin{signalflow}[node distance=15mm]
        % Elemente
        \tikzgrid{
            \node[input]    (in)    {}          &
            \node[adder]    (a1)    {}          &
            \node[delay]    (d1)    {Regler}    & &
            \node[delay]    (d2)    {Strecke}   &  
            \node[node]     (n1)    {}          &
            \node[output]   (out)   {}
        }
        % Pfade
        \path[r>] (in) -- node[above]{$w=\sigma(t)$} (a1);
        \path[r>] (a1) -- node[above]{$e$} (d1);
        \path[r>] (d1) -- node[above]{$x$} (d2);
        \path[r>] (d2) -- node[above]{$y$} (out);
        \path[r>] (n1) -- ++(0,-0.5)  -| node[left, pos=0.95] {$-$} (a1);
        \path[r>] (d2) ++(0,1) -- node[right]{$z=0$} (d2); 
    \end{signalflow}
\end{figure}
%
\subsection{Störungsverhalten}


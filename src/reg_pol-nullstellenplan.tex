\section{Pol-Nullstellenplan}
Betrachtet man eine Übertragungsfunktion als Quotient in der Form
\[  
    G(s) = \frac{Z(s)}{N(s)}
\]
so können Nullstellen sowohl für das Zählerpolynom $Z(s)$ als auch 
für das Nennerpolynom $N(s)$ ermittelt werden. Nullstellen des 
Zählerpolynoms sind zugleich Nullstellen der Übertragungsfunktion.
Die Nullstellen des Nennerpolynoms werden als Polstellen der
Übertragungsfunktion bezeichnet.
%
\begin{figure}[h!]
    \centering
    \begin{tikzpicture}[scale=1]
        % Koordinatensystem
        \draw[->] (-2,0) -- (2,0) node[below] {$Re$};
        \draw[->] (0,-2) -- (0,2) node[left] {$Im$}; 
        % Einheitskreis
        \draw[dotted] (0,0) circle (1);
        % Nullstellen
        \draw[red, thick] (0.5,0) circle (0.1);
        % Pole
        \node[cross out, draw=red, thick] at (-1,1) {};
        \node[cross out, draw=red, thick] at (-1,-1) {};
    \end{tikzpicture}
\end{figure}
%
Die folgenden Regeln sind in der Grafik angedeutet.
\begin{itemize}
    \item Je grösser der imaginäre Anteil einer Polstelle wird, desto
        höher wird die Frequenz der Sprungantwort.
    \item je grösser der reelle Anteil einer Polstelle wird, desto
        steiler wird die Sprungantwort.
    \item Polstellen mit negativen Realteil sind stabil.
    \item Polstellen mit einem Realteil von $\geq 0$ sind instabil.
\end{itemize}
%
Anhand der Polstellen lässt sich ein Übertragungsglied oder ein System 
bezüglich dessen Verhalten analysieren. Beispielsweise kann direkt darauf
geschlossen werden, in welche Weise die Sprungantwort auszusehen hat und
wie die Stabilität zu bewerten ist.
%
\begin{figure}[h!]
    \centering
    \begin{tikzpicture}[scale=2]
        % Rasterübersicht %%%%%%%
        %   a1  b1  c1  d1  e1  %
        %   a2  b2  c2  d2  e2  %
        %   a3  b3  c3  d3  e3  %
        %%%%%%%%%%%%%%%%%%%%%%%%% 
        % Koordinaten
        \coordinate (a1) at (-1.5,  0.9);
        \coordinate (a2) at (-1.5,  0.1);
        \coordinate (a3) at (-1.5, -0.7);
        %
        \coordinate (b1) at (-0.7,  0.9);
        \coordinate (b2) at (-0.7,  0.1);
        \coordinate (b3) at (-0.7, -0.7);
        %
        \coordinate (c1) at ( 0.1,  0.9);
        \coordinate (c2) at ( 0.1,  0.1);
        \coordinate (c3) at ( 0.1, -0.7);
        %
        \coordinate (d1) at ( 0.9,  0.9);
        \coordinate (d2) at ( 0.9,  0.1);
        \coordinate (d3) at ( 0.9, -0.7);
        %
        \coordinate (e1) at ( 1.7,  0.9);
        \coordinate (e2) at ( 1.7,  0.1);
        \coordinate (e3) at ( 1.7, -0.7);
        % Koordinatensystem
        \draw[->] (-2,0) -- (2.5,0) node[below] {$Re$};
        \draw[->] (0,-1.5) -- (0,2) node[left] {$Im$}; 
        % Einheitskreis
        %\draw[dotted] (0,0) circle (1);
        % 
        \begin{scope}[shift={(a1)}]
            \draw[->] (0,0) -- (0.5,0);
            \draw[->] (0,0) -- (0,0.5);
            \node[cross out, draw=black, thick] at (-0.1,-0.1) {};
        \end{scope}
        \begin{scope}[shift={(b1)}]
            \draw[->] (0,0) -- (0.5,0);
            \draw[->] (0,0) -- (0,0.5);
            \node[cross out, draw=black, thick] at (-0.1,-0.1) {};
        \end{scope}
        \begin{scope}[shift={(c1)}]
            \draw[->] (0,0) -- (0.5,0);
            \draw[->] (0,0) -- (0,0.5);
            \draw[blue, thick] 
                (0,0) sin (0.1,0.4) 
                    cos (0.15,0.2) 
                    sin (0.2,0)
                    cos (0.25,0.2)
                    sin (0.3,0.4)
                    cos (0.35,0.2)
                    sin (0.4,0.0)
                    cos (0.45,0.2);
            \node[cross out, draw=black, thick] at (-0.1,-0.1) {};
        \end{scope}
        \begin{scope}[shift={(d1)}]
            \draw[->] (0,0) -- (0.5,0);
            \draw[->] (0,0) -- (0,0.5);
            \node[cross out, draw=black, thick] at (-0.1,-0.1) {};
        \end{scope}
        \begin{scope}[shift={(e1)}]
            \draw[->] (0,0) -- (0.5,0);
            \draw[->] (0,0) -- (0,0.5);
            \node[cross out, draw=black, thick] at (-0.1,-0.1) {};
        \end{scope}
        %
        \begin{scope}[shift={(a2)}]
            \draw[->] (0,0) -- (0.5,0);
            \draw[->] (0,0) -- (0,0.5);
            \draw[green, thick] (0.5,0.5) parabola bend (0.2,0.5) (0,0);
            \node[cross out, draw=black, thick] at (-0.1,-0.1) {};
        \end{scope}
        \begin{scope}[shift={(b2)}]
            \draw[->] (0,0) -- (0.5,0);
            \draw[->] (0,0) -- (0,0.5);
            \draw[green, thick] (0.5,0.5) parabola bend (0.5,0.5) (0,0);
            \node[cross out, draw=black, thick] at (-0.1,-0.1) {};
        \end{scope}
        \begin{scope}[shift={(c2)}]
            \draw[->] (0,0) -- (0.5,0);
            \draw[->] (0,0) -- (0,0.5);
            \draw[blue, thick] (0,0) -- (0.5,0.5);
            \node[cross out, draw=black, thick] at (-0.1,-0.1) {};
        \end{scope}
        \begin{scope}[shift={(d2)}]
            \draw[->] (0,0) -- (0.5,0);
            \draw[->] (0,0) -- (0,0.5);
            \draw[red, thick] (0,0) parabola bend (0,0) (0.5,0.5);
            \node[cross out, draw=black, thick] at (-0.1,-0.1) {};
        \end{scope}
        \begin{scope}[shift={(e2)}]
            \draw[->] (0,0) -- (0.5,0);
            \draw[->] (0,0) -- (0,0.5);
            \draw[red, thick] (0,0) parabola bend (0,0) (0.2,0.5);
            \node[cross out, draw=black, thick] at (-0.1,-0.1) {};
        \end{scope}
        %
        \begin{scope}[shift={(a3)}]
            \draw[->] (0,0) -- (0.5,0);
            \draw[->] (0,0) -- (0,0.5);
            \node[cross out, draw=black, thick] at (-0.1,-0.1) {};
        \end{scope}
        \begin{scope}[shift={(b3)}]
            \draw[->] (0,0) -- (0.5,0);
            \draw[->] (0,0) -- (0,0.5);
            \node[cross out, draw=black, thick] at (-0.1,-0.1) {};
        \end{scope}
        \begin{scope}[shift={(c3)}]
            \draw[->] (0,0) -- (0.5,0);
            \draw[->] (0,0) -- (0,0.5);
            \draw[blue, thick] 
                (0,0) sin (0.1,0.4) 
                    cos (0.15,0.2) 
                    sin (0.2,0)
                    cos (0.25,0.2)
                    sin (0.3,0.4)
                    cos (0.35,0.2)
                    sin (0.4,0.0)
                    cos (0.45,0.2);
            \node[cross out, draw=black, thick] at (-0.1,-0.1) {};
        \end{scope}
        \begin{scope}[shift={(d3)}]
            \draw[->] (0,0) -- (0.5,0);
            \draw[->] (0,0) -- (0,0.5);
            \node[cross out, draw=black, thick] at (-0.1,-0.1) {};
        \end{scope}
        \begin{scope}[shift={(e3)}]
            \draw[->] (0,0) -- (0.5,0);
            \draw[->] (0,0) -- (0,0.5);
            \node[cross out, draw=black, thick] at (-0.1,-0.1) {};
        \end{scope}
    \end{tikzpicture}
\end{figure}


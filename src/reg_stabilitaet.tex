\section{Stabilität}
In der Regelungstechnik gibt es eine allgemeine Definition für die
Stabilität. Diese besagt, dass ein System stabil ist, wenn sich die
Sprungantwort einem Endwert annähert. Hierzu muss also ein Grenzwert
existieren.
\[ 
    \exists\, \lim_{t \rightarrow \infty} h(t)
\]

\subsection{BIBO-Stabilität}
Die BIBO-Stabilität (\textit{engl. ''bounded input - bounded output``}) 
bezeichnet eine Stabilität der Art, dass eine beschränkte Eingangsgrösse 
eine beschränkte Ausgangsgrösse zur Folge hat. 
\[  
    \int\limits_{0}^{\infty} \lvert g(t) \rvert \, dt < 0
\]

\subsection{Hurwitz}
Das Hurwitzkriterium bestimmt eindeutig die Stabilitätseigenschaft einer
Übertragungsfunktion. Eine solche ist genau dann stabil, wenn das 
Nennerpolynom $N(s)$ der Übertragungsfunktion 
\[  
    G(s) 
        = \frac{Z(s)}{N(s)} 
        = \frac{Z(s)}{a_0s^0 + a_1s^1 + a_2s^2 + \dots + a_ns^n} 
\]
die folgenden Kriterien erfüllt.
%
\begin{enumerate}
    \item Alle Koeffizenten des Polynoms für den gegebenen Grad sind 
        vorhanden. \[ a_n \neq 0 \]
    \item Alle Koeffizienten des Polynoms sind grösser als null (also
        positiv). \[ a_n > 0 \]
    \item Die Determinante und alle Unterdeterminanten sind grösser 
        als null. \[ \Delta_i > 0 \]
\end{enumerate}
%
Für Polynome vom Grad $n \leq 2$ genügt die Prüfung der ersten beiden 
Kriterien.

\subsubsection{Determinanten berechnen}
Die Determinante eines Polynoms 
\[  
    N(s) =  a_0s^0 + a_1s^1 + a_2s^2 + \dots + a_ns^n
\]
lautet
\[ 
    \Delta_n =
       \begin{vmatrix}
            a_1 & a_3 & a_5 & a_7 & \dots \\
            a_0 & a_2 & a_4 & a_6 & \dots \\
            0   & a_1 & a_3 & a_5 & \dots \\
            0   & a_0 & a_2 & a_4 & \dots \\
            0   & 0   & a_1 & a_3 & \dots \\
            0   & 0   & a_0 & a_2 & \dots \\
            \vdots & \vdots & \vdots & \vdots & \ddots \\
        \end{vmatrix}
\]
Die Wahl der Kantengrösse hängt vom Grad des betrachteten Polynoms ab.
Hat man ein Polynom von Grad $n$, dann muss die Matrix auch $n$ Zeilen
und $n$ Spalten haben. 

Die Unterdeterminanten dieser Determinante sind dann all jene Matrizen,
welche eine kleinere Kantenlänge haben. Diese sind allesamt ausgehend
von der linken oberen Ecke anzusetzen.
\[  
    \Delta_1 =
        \begin{vmatrix}
            a_1 \\
        \end{vmatrix}
\]
\[  
    \Delta_2 =
        \begin{vmatrix}
            a_1 & a_3 \\
            a_0 & a_2 \\
        \end{vmatrix}
\]
\[  
    \Delta_3 =
        \begin{vmatrix}
            a_1 & a_3 & a_5 \\
            a_0 & a_2 & a_4 \\
            0   & a_1 & a_3 \\
        \end{vmatrix}
\]
Die Berechnung der Determinaten kann mittels zweier Sätze erfolgen:
\begin{itemize}
    \item Laplace'scher Entwicklungssatz (rekursiv)
    \item Regel von Sarrus
\end{itemize}

\subsubsection{Laplace'scher Entwicklungssatz}
Mit dem Laplace'schen Entwicklungssatz lassen sich Determinanten
resukrsiv berechnen. Hierzu kann nach Spalten oder Zeilen vorgegangen
werden. Das folgenden Beispiel zeigt das Verahren nach Zeilen. 

Man nimmt das erste Element $a$ und multipliziert es mit der Determinaten
der Untermatrix. Dann subtrahiert man das Produkt des nächsten Elements 
$b$ und der Determinante von dessen Untermatrix usw. 

Bei diesem Vorgang ist zu beachten, dass das Vorzeichen zwischen den
Summanden alterniert. 
\[  
    det(A)
        =   \begin{vmatrix}
                a & b & c \\
                d & e & f \\
                g & h & i \\
            \end{vmatrix}
        = a \cdot 
            \begin{vmatrix} 
                e & f \\
                h & i \\
            \end{vmatrix}
            - 
            d \cdot 
            \begin{vmatrix}
                b & c \\
                h & i \\
            \end{vmatrix}
            + 
            g \cdot 
            \begin{vmatrix}
                b & c \\
                e & f \\
            \end{vmatrix} 
\]

\subsubsection{Regel von Sarrus}
Die Regel von Sarrus ermöglicht es die Determinate einer Matrix direkt
auszudrücken. Hierzu wird jeweils diagonal durch die Matrix ein Produkt
aufgestellt. Alle Produkte gleichgerichteter Diagonalen werden 
aufsummiert, wobei die eine Richtung positiv und die andere negativ
belegt wird.
\[  
    det(A)
        =   \begin{vmatrix}
                a & b & c \\
                d & e & f \\
                g & h & i \\
            \end{vmatrix}
        = (aei + dhc + gbf) - (ceg + fha + ibd)
\]


\subsection{Bode}


\subsection{Nyquist}
Das Nyquist-Kriterium wird unterteilt in zwei Anwendungsfälle.
\begin{itemize}
    \item das allgemeine Nyquist-Kriterium (am geschlossenen Regelkreis)
    \item das spezielle Nyquist-kriterium (am offenen Regelkreis)
\end{itemize}

\subsubsection{Allgemeines Nyquist-Kriterium}
Das allgemeine Nyquist-Kriterium prüft die Stabilität des geschlossenen
Regekreises anhand der Pole. 
%
\begin{figure}[h!]
    \begin{signalflow}[node distance=15mm]
        % Elemente
        \tikzgrid{
            \node[input]    (in)    {}          &
            \node[adder]    (a1)    {}          &
            \node[delay]    (d1)    {$G_0$}     &
            \node[node]     (n1)    {}          &
            \node[output]   (out)   {}
        }
        % Pfade
        \path[r>] (in) -- (a1);
        \path[r>] (a1) -- (d1);
        \path[r>] (d1) -- (out);
        \path[r>] (n1) -- ++(0,-0.5) -| node[left, pos=0.95] {$-$} (a1);
    \end{signalflow}
\end{figure}
%
Um dies zu prüfen, müssen zuerst die 
Parameter $a$ und $r$ bestimmt werden. $a$ bezeichnet die Anzhal der Pole
auf der imaginären Achse und $r$ die Anzhal der Pole rechtsseitig der
imaginären Achse. Ist dann die folgende Bedingung 
\[  
    \Delta \varphi = a \frac{\pi}{2} + r \pi
\]
mit diesen Parametern erfüllt, wobei $\Delta \varphi$ den resultierenden 
Winkel bezeichnet, dann ist $G_0$ stabil.


\subsubsection{Spezielles Nyquist-Kriterium}
Dasi spezielle Nyquist-Kriterium untersucht die Stabilität anhand der
Phasenreserve des offenen Regelkreis. 
Hierzu wird einfach die Rückführung getrennt. So wandelt sich die 
Übertragungsfunktion zu
\[  
    G(s) = \frac{G_0}{1+G_0} \Rightarrow G_{\text{offen}}(s) = G_0
\]
%
\begin{figure}[h!]
    \begin{signalflow}[node distance=15mm]
        % Elemente
        \tikzgrid{
            \node[input]    (in)    {}          &
            \node[adder]    (a1)    {}          &
            \node[delay]    (d1)    {$G_0$}     &
            \node[node]     (n1)    {}          &
            \node[output]   (out)   {}
        }
        % Pfade
        \path[r>] (in) -- (a1);
        \path[r>] (a1) -- (d1);
        \path[r>] (d1) -- (out);
        \coordinate (p1) at ($ (n1) + (-1,-0.5) $);
        \coordinate (p2) at ($ (p1) + (-0.2,0) $);
        \draw[thick] (n1) |- (p1) -- ++(-0.2,-0.2);
        \path[r>] (p2) -| node[left, pos=0.95] {$-$} (a1);
    \end{signalflow}
\end{figure}
%

\subsection{Phasenreserve}

\begin{figure}[h!]
    \centering
    \begin{tikzpicture}[scale=3]
        \draw[->] (-1.5,0) -- (1.5,0) node[below] {$Re$};
        \draw[->] (0,-1.5) -- (0,1.5) node[left] {$Im$};
        \draw[thick] 
            (-1,-0.1) -- (-1,0.1) node[above left] {$-1$}; 
        \draw[dotted, name path=kreis] (0,0) circle (1);
        \draw[name path=kurve] (-1.1,-1.5) 
            to [out=80, in=160] (-0.2,0.4)
            to [out=-20, in=90] (0,0);
        \draw[name intersections={of=kreis and kurve, name=i}]
            (i-1) -- (0,0);
    \end{tikzpicture}
\end{figure}

\subsection{Amplitudenreserve}

\subsection{Totzeitreserve}

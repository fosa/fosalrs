% coding:utf-8

%FOSALRS, a LaTeX-Code for a electrical summary of control theory
%Copyright (C) 2013, Daniel Winz, Ervin Mazlagic

%This program is free software; you can redistribute it and/or
%modify it under the terms of the GNU General Public License
%as published by the Free Software Foundation; either version 2
%of the License, or (at your option) any later version.

%This program is distributed in the hope that it will be useful,
%but WITHOUT ANY WARRANTY; without even the implied warranty of
%MERCHANTABILITY or FITNESS FOR A PARTICULAR PURPOSE.  See the
%GNU General Public License for more details.
%----------------------------------------

\chapter{Systeme}
\newpage

\section{LZI}

\subsection{Linearität}

\subsubsection{Überlagerungsprinzip}
\[
    y_1 = f(u_1)
\]
\[
    y_2 = f(u_2)
\]
\[
    f(u_1 + u_2) = y_1 + y_2
\]

\subsubsection{Verstärkungsprinzip}
\[
    y = f(u)
\]
\[
    f(k \cdot u) = k \cdot y
\]

\subsection{Zeitinvarianz}
\[
    y = f(u) \neq f(t)
\]

\clearpage
\section{Glieder}


\subsection{$P$}
DGL: 
\[
    x_a = K_P \cdot x_e
\]
Übertragungsfunktion: 
\[
    G(s) = K_P
\]
Sprungantwort: 
\[
    h(t) = K_P
\]
\begin{figure}[h!]
    \begin{tikzpicture}
        \draw[->] (-0.2,0) -- (5,0);
        \draw[->] (0,-0.2) -- (0,2.2);
        \draw[thick, red] (-0.2,0) -- (0,0) -- (0,2) -- (5,2);
        \draw[-] (-0.2,2) node[left] {$K_P$} -- (0.2,2);
    \end{tikzpicture}
\end{figure}
\FloatBarrier
\noindent
Pol-Nullstellen-Plan: 
\begin{figure}[h!]
    \begin{tikzpicture}
        \draw[->] (-2,0) -- (2,0) node[above right] {$Re$};
        \draw[->] (0,-2) -- (0,2) node[above right] {$Im$};
    \end{tikzpicture}
\end{figure}
\clearpage
\noindent
Bode-Diagramm: 
\begin{figure}[h!]
    \begin{tikzpicture}
        \draw[->] (-0.2,0) -- (5.2,0) node[above right] {$\omega$};
        \draw[->] (0,-2) -- (0,1.2) node[above right] {$|g|$};
        \draw[thick, red] (0,0.5) node[left] {$K_{P_{dB}}$} -- (5,0.5);
    \end{tikzpicture}
\end{figure}
\begin{figure}[h!]
    \begin{tikzpicture}
        \draw[->] (-0.2,0) -- (5.2,0) node[above right] {$\omega$};
        \draw[->] (0,-2) -- (0,1.2) node[above right] {$\varphi$};
        \draw[thick, red] (0,0) -- (5,0);
    \end{tikzpicture}
\end{figure}
\FloatBarrier
\noindent
Ortskurve: 
\begin{figure}[h!]
    \begin{tikzpicture}
        \draw[->] (-2,0) -- (2,0) node[above right] {$Re$};
        \draw[->] (0,-2) -- (0,2) node[above right] {$Im$};
        \draw[->, thick, red] (0,0) -- (1.5,0) node[above] {$K_P$};
    \end{tikzpicture}
\end{figure}

\clearpage
\subsection{$PT_1$}
DGL:
\[
    x_a = K_P \cdot x_e - T_1 \cdot \dot{x}
\]
Übertragungsfunktion:
\[
    G(s) = \frac{K_P}{1 + s T_1}
\]


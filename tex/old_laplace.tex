% coding:utf-8

%FOSALRS, a LaTeX-Code for a electrical summary of control theory
%Copyright (C) 2013, Daniel Winz, Ervin Mazlagic

%This program is free software; you can redistribute it and/or
%modify it under the terms of the GNU General Public License
%as published by the Free Software Foundation; either version 2
%of the License, or (at your option) any later version.

%This program is distributed in the hope that it will be useful,
%but WITHOUT ANY WARRANTY; without even the implied warranty of
%MERCHANTABILITY or FITNESS FOR A PARTICULAR PURPOSE.  See the
%GNU General Public License for more details.
%----------------------------------------

\chapter{Laplace Transformation}

\section{Laplace-Transformation}
Sei $u(t)$ ein Signal. \\
Falls das Integral 
\[ U(s) = \int\limits_{0}^{\infty} u(t) \cdot e^{-st} ~ dt \]
konvergiert, so heisst 
\[ \boxed{\mathcal{L} \lbrace u(t) \rbrace = U(s)} \]
die Laplace-Transformierte von $u(t)$. \\
Erklärungen: 
\begin{itemize}
  \item $s \in \mathbb{C}$
  \item $\int\limits_{0^-}^{\infty} u(t) \cdot e^{-st} ~ dt 
  = \lim_{a \nearrow 0} \int\limits_{a}^{\infty} u(t) \cdot e^{-st} ~ dt $
  \item Falls $u(t)$ sprungstetig in $t = 0$ ist, kann man$0^-$ durch $0$ 
        ersetzen. \\
        $u(t)$ heisst sprungstetig, wenn es nur endliche Sprünge macht. 
  \item Wenn $u(t)$ sprungstetig ist und höchstens exponentiell wächst, 
        dann konvergiert das Integral für alle $s \in \mathbb{C}$ mit 
        $\text{Re}(s) > \sigma_0$\\
        % >>> Grafik Konvergenzbereich
\end{itemize}
\[ s = \sigma + \omega \cdot j, \qquad Re(s) = \sigma 
\qquad \sigma, \omega \in \mathbb{R} \]
Die Menge 
\[ KB = \lbrace s \in \mathbb{C} | (*) konvergiert \rbrace 
= \lbrace s \in \mathbb{R} | \text{Re}(s) > \sigma_0 \rbrace \]
heisst Konvergenzbereich. \\
Bsp: 
\[ u(t) = \sigma(t) \]
\[ \mathcal{L} \lbrace \sigma(t) \rbrace 
= \int\limits_{0^-}^{\infty} \underbrace{\sigma(t) \cdot e^{-st}}_
{\text{sprungstetig in }t = 0} ~ dt = \int\limits_{0}^{\infty} e^{-st} ~ dt 
= \left. -\frac{1}{s} e^{-st} \right|_{t = 0}^{t = \infty} 
= -\frac{1}{s} \underbrace{\lim_{t \to \infty} \cdot e^{-st}}_
{0 \text{ falls Re}(s) > 0} + \frac{1}{s} \]
Also
\[ \mathcal{L} \lbrace \sigma(t) \rbrace = \frac{1}{s} \]
Andere Notation: 
\[ \sigma(t) ~ \laplace ~ \frac{1}{s} \qquad \text{KB: Re}(s) > 0\]




\[ \boxed{\mathcal{L} \lbrace u(t) \rbrace 
= \int\limits_{0}^{\infty} u(t) \cdot e^{-st} ~ dt \qquad , s \in \mathbb{C} 
= U(s)} \]
\[ \boxed{\begin{array}{rcl}
\text{Zeitbereich} &  & \text{Bildbereich} \\
u(t) & \laplace & U(s)
\end{array}} \]


\subsection{Eigenschaften der Laplace-Transformation}

\subsubsection{Linearität}
\[ \boxed{\mathcal{L}\lbrace c \cdot u(t) \rbrace 
= c \cdot \mathcal{L}\lbrace u(t) \rbrace} \]
\[ \boxed{\mathcal{L}\lbrace u_1(t) \cdot u_2(t) \rbrace 
= \mathcal{L}\lbrace u_1(t) \rbrace + \mathcal{L}\lbrace u_2(t) \rbrace} \]

\subsubsection{Differentiationssatz}
\[ \boxed{\mathcal{L}\lbrace \dot{u}(t) \rbrace = s \cdot U(s) - u(0^-)} \]
\[ \boxed{\dot{u}(t) ~ \laplace ~ s \cdot U(s) - u(0^-)} \]
\[ \boxed{\dddot{u}(t) ~ \laplace ~ s^3 \cdot U(s) - s^2 \cdot u(0^-) 
- s \cdot \dot{u}(0^-) - \ddot{u}(0^-)} \]
Häufig: Anfangsbedingungen $= 0$. 
Dann: 
\[ \boxed{\frac{d^n}{dt^n} u(t) ~ \laplace ~ s^n \cdot U(S)} \]

\subsection{Integrationssatz}
\[ \boxed{\int\limits_{0^-}^{t} u(t) ~ dt ~ \laplace ~ \frac{1}{s} \cdot U(s)} \]

\subsection{Zeitverschiebungssatz}
Ein Signal u(t) heisst kausal, wenn $u(t) = 0$ für $t < 0$. 
\[ \boxed{u(t - t_0) \cdot \sigma(t - t_0) ~ \laplace ~ e^{-t_0s} \cdot U(s)} \]

\subsection{Laplace Transformation einiger Grundfunktionen}
\[ \begin{array}{ccc}
u(t) & \laplace & U(s) \\\\
\delta(t) & 
    & 1 \\\\
\sigma(t) & 
    & \frac{1}{s} \\\\
t \cdot \sigma(t) & 
    & \frac{1}{s^2} \\\\
t^2 \cdot \sigma(t) & 
    & \frac{2}{s^3} \\\\
\vdots & 
    & \vdots \\\\
e^{a \cdot t} & 
    & \frac{1}{s - a} \\\\
\sin(t) & 
    & \frac{1}{s^2 + 1} \\\\
\cos(t) & 
    & \frac{s}{s^2 + 1} \\\\
\end{array} \]

\section{Inverse Laplace Transformation}
\begin{enumerate}
  \item Laplace Transformation anwenden\\
        Wende $\mathcal{L}$ auf $(*)$ an und schreibe 
        $U(s) = \mathcal{L} \{ u(t) \}$ \\
        $V(s) = \mathcal{L} \{ v(t) \}$ \\
        \[\mathcal{L} \{ \dot{v} + \frac{1}{RC} \cdot v \} 
        = \mathcal{L} \{ \dot{u} \}\]
        Linearität
        \[\mathcal{L} \{ \dot{v} \} + \frac{1}{RC} \cdot \mathcal{L} \{ v \} 
        = \mathcal{L} \{ \dot{u} \}\]
        Diff-Satz
        \[s \mathcal{L} \{ \dot{v} \} - v(0^-) 
        + \frac{1}{RC} \cdot \mathcal{L} \{ v \} 
        = s \mathcal{L} \{ \dot{u} \} - u(0^-) \]
        \[ s V + \frac{1}{RC} V = s U - u(0^-) \]
        \[ \boxed{V = \frac{s U - u(0^-)}{s + \frac{1}{RC}}} \]
        Das ist die Beschreibung des Systems im Bildraum für jedes 
        Eingabesignal $u(t)$. 
  \item Eingabesignal transformieren und einsetzen
        \[ u(t) = \sigma(t) \Rightarrow U(s) = \frac{1}{s} \qquad u(0^-) = 0 \]
        \[ \Rightarrow V = \frac{s \cdot \frac{1}{s} - 0}{s + \frac{1}{RC}} \]
        \[ \boxed{V(s) = \frac{1}{s + \frac{1}{RC}}} 
        \quad \text{Lösung im Bildbereich} \]
  \item Zurücktransformieren\\
        Finde $v(t)$ mit 
        \[ v(t) ~ \laplace ~ \frac{1}{s + \frac{1}{RC}} \]
        Tabelle: 
        \[ e^{at}\sigma(t) ~ \laplace ~ \frac{1}{s - a} \]
        Also: 
        \[ v(t) = e^{-\frac{1}{RC} t} \sigma(t) \]
\end{enumerate} 
Schritt 3 ist die Berechnung der inversen Laplace Transformation

\subsection{Eindeutigkeitssatz}
\[ U(s) = V(s) \]
\[ u(t) = v(t) \forall t \geq 0 \]
Notation: 
\[ \mathcal{L} \{ u(t) \} = U(s) \]
\[ u(t) = \mathcal{L}^{-1} \{ U(s) \} \]

\subsection{Dämpfungssatz / Frequenzverschiebungssatz}
\[ e^{at} \cdot u(t) ~ \laplace ~ U(s - a) \]
Vorsicht! Vorzeichen anders als beim Zeitverschiebungssatz. 

\subsection{Partialbruchzerlegung}
Erinnerung: \\
\begin{itemize}
  \item Rationale Funktion
        \[ R(s) = \frac{Z(s)}{N(s)} \]
        mit Polynomen $Z(s), N(s)$
  \item Polstellen = Definitionlücken = Nullstellen von $N(s)$
  \item Faktorisierung
        \[ N(s) = (s - s_1) \cdot (s - s_2) \cdot \ldots \cdot (s - s_n) \]
        wobei $s_1, \ldots, s_n$ alle Nullstellen von $N(s)$ 
        I.a. $s_i \in \mathbb{C}$
  \item Vielfachheiten\\
        Sind mehrere $s_i$ gleich, so nennt man die Anzahl der gleichen $s_i$ 
        die Vielfachheit der Polstelle $s_i$. 
\end{itemize}
\begin{enumerate}
  \item Einfache Polstellen \\
        Annahme: Alle Polstellen haben Vielfachheit 1
        \[ R(s) = \frac{Z(s)}{(s - s_1) \cdot \ldots \cdot (s - s_n} \]
        Wobei $s_1, \ldots, s_n$ paarweise verschieden \\
        Annahme: Grad von Z(s) < n \\
        Dann: 
        \[ R(s) = \frac{a_1}{s - s_1} + \frac{a_2}{s - s_2} + \ldots 
        + \frac{a_n}{s - s_2} \]
        mit Konstanten $a_1, \ldots, a_n$\\
        Berechnung der $a_i$: 
        \[ a_i = \lbrack R(s) (s - s_i \rbrack_{s = s_i} \]
  \item Mehrfache Polstellen
        \[ R(s) = \frac{Z(s)}{(s - a)^n} \]
        Dann: 
        \[ R(s) = \frac{a1}{s - a} + \frac{a_2}{(s - a)^2} + \ldots 
        + \frac{a_n}{(s - a)^n} \]
\end{enumerate}

\section{Grenzwertsätze}

\subsection{Anfangswertsatz}
Sei $u(t)$ an der Stelle $t = 0$ sprungstetig. Dann gilt 
\[ u(0^+) = \lim\limits_{s \to \infty} s \cdot U(s) \]
Bemerkung: \\
Wenn $U(s) = \dfrac{Z(s)}{N(s)}$ eine rationale Funktion mit 
$\text{Grad}(Z(s)) < \text{Grad}(N(s)$, dann ist u(t) sprungstetig

\subsection{Endwertsatz}
Sein $u(t)$ ein Signal, für das der Endwert 
$u(\infty) = \lim\limits_{t \to \infty} u(t)$ existiert. \\
Dann gilt: 
\[ u(\infty) = \lim\limits_{s \to 0} s \cdot U(s) \]
Kriterium: \\
Wenn alle Polstellen von $s \cdot U(s)$ links von der imaginären Achse liegen, 
dann existiert der Endwert $u(\infty)$. \\
Mit anderen Worten: Der Realteil aller Nullstellen muss negativ sein. 

\section{Lösen von DGL}
Beispiel:
\[ \mathcal{L} \{ R C \dot{v} + v = u \} \]
\[ v(0^-) = v_0 \]
\begin{enumerate}
  \item Laplace Transformation anwenden
        \[ U(s) = \mathcal{L} \{ u(t) \} \qquad V(s) = \mathcal{L} \{ v(t) \} \]
        Laplace Transformation
        \[ \mathcal{L} \{ R C \dot{v} + v \} = \mathcal{L} \{ u \} \]
        Linearität
        \[ R C \cdot \mathcal{L} \{ \dot{v} \} + \mathcal{L} \{ v \} 
        = \mathcal{L} \{ u \} \]
        Differentiationssatz
        \[ R C \cdot ( s \cdot V(s) - v_0 )  + V(s) = U(s) \]
  \item Auflösen nach $V(s)$
        \[ \boxed{V(s) = \frac{U(s) + R \cdot C}{R \cdot C \cdot s + 1}} \]
  \item Eingangssignal transformieren und einsetzen
        \[ u(t) = 2 \cdot \sigma(t) \cdot v_0 = 0 \]
        \[ U(s) = \frac{2}{s} \]
        \[ V(s) = \frac{\frac{2}{s} + RC}{RC \cdot s + 1} \]
  \item Rücktransformation
        \[ V(s) = \frac{2}{s (RC \cdot s + 1)}  + \frac{RC}{RC \cdot s + 1} \]
        Eintrag 41 in grosser Tabelle: 
        \[ \boxed{1 - e^{-\frac{t}{a}} ~ \laplace ~ \frac{1}{s(1 + a s}} 
        \qquad \text{Hier mit }a = RC \]
        1. Term: 
        \[ \mathcal{L}^{-1} \{ \frac{2}{s(RC \cdot s + 1} \} 
        = 2 \cdot (1 - e^{-\frac{t}{RC}}) \]
        2. Term: \\
        Eintrag 36
        \[ \boxed{\frac{1}{a} e^{-\frac{a}{t}} ~ \laplace ~ \frac{1}{1 + a s}} 
        \qquad \text{Hier mit }a = RC \]
        \[ \mathcal{L}^{-1} \{ \frac{RC}{RC \cdot s} \} 
        = RC \cdot \frac{1}{RC} \cdot e^{-\frac{t}{RC}} = e^{-\frac{t}{RC}} \]
        Resultat: 
        \[ v(t) = 2 (1 - e^{-\frac{t}{RC}}) + e^{-\frac{t}{RC}} 
        = 2 - e^{-\frac{t}{RC}} \qquad \text{für }t \geq 0 \]
        Annahme: 
        \[ v(t) = const \forall t < 0 \]
        AW: 
        \[ v(0^-) = 1 \Rightarrow v(t) = 1 \forall t < 0 \] 
\end{enumerate}

\section{Übergangsgleichung und spezielle Antworten}
Beispiel von oben mit AW $v_0 = 0$
\[ V(s) = \underbrace{\frac{1}{RC \cdot s + 1}}_{G(s)} \cdot U(s) \]
$G(s)$: Übertragungsfunktion\\
Die Zeitfunktion $g(t)$ von $G(s)$ heisst Gewichtsfunktion\\
Bemerkung: Ist ein System durch eine lineare DGL mit konstanten Koeffizienten 
beschrieben, dann ist das System LZI \emph{nur}, wenn alle Anfangsbedingungen
\emph{null} sind. \\
d.h. 
\[ u(0^-) = \dot{u}(0^-) = \ddot{u}(0^-) = \ldots = 0 \]
\[ v(0^-) = \dot{v}(0^-) = \ddot{v}(0^-) = \ldots = 0 \]
Nur dann wird das System durch eine Übertragungsfunktion beschreiben. \\
Theorem: 
Ein System ist genau dann LZI, wenn sein Übertragungsverhalten sich durch 
\[ V(s) = G(s) \cdot U(s) \]
beschreiben lässt, wobei $G(s)$ eine feste Funktion ist. Diese nennen wir 
Übertragungsfunktion. 

\section{Faltung}
Definition: Die Faltung zweier Signale $u(t)$ und $v(t)$ ist 
\[ (u * v)(t) = \int\limits_{-\infty}^{\infty} u(\tau) v(t - \tau) ~ d \tau \]
Bemerkung: Sind die Signale $u(t)$ und $v(t)$ sprungstetig und kausal, dann 
gilt: 
\[ (u * v)(t) = \int\limits_{0}^{t} u(\tau) v(t - \tau) ~ d \tau \]
Faltungssatz: 
Sind $u(t), v(t)$ kausal, so gilt: 
\[ (u * v) (t) ~ \laplace ~ U(s) \cdot V(s) \]

\subsection{Impulsantwort}
Ist $g(t)$ die Gewichtsfunktion eines LZI Systems, so gilt
\[ v(t) = (g * u)(t) \]
Fazit: Impulsantwort = Gewichtsfunktion

\subsection{Sprungantwort}
Definition: Sprungantwort = Antwort auf $\sigma(t) = h(t)$ 
(AW nicht unbedingt 0) \\
Bemerkung: Bei LZI Systemen: 
\[ h(t) = \int\limits_{0}^{t} g(\tau) ~d \tau \]
(folgt aus Integrationssatz)

\section{Stabilität}
In der Regelungstechnik gibt es eine allgemeine Definition für die
Stabilität. Diese besagt, dass ein System stabil ist, wenn sich die
Sprungantwort einem Endwert annähert. Hierzu muss also ein Grenzwert
existieren.
\[ 
    \exists\, \lim_{t \rightarrow \infty} h(t)
\]

\subsection{BIBO-Stabilität}
Die BIBO-Stabilität (\textit{engl. ''bounded input - bounded output``}) 
bezeichnet eine Stabilität der Art, dass eine beschränkte Eingangsgrösse 
eine beschränkte Ausgangsgrösse zur Folge hat. 
\[  
    \int\limits_{0}^{\infty} \lvert g(t) \rvert \, dt < 0
\]

\subsection{Hurwitz}
Das Hurwitzkriterium bestimmt eindeutig die Stabilitätseigenschaft einer
Übertragungsfunktion. Eine solche ist genau dann stabil, wenn das 
Nennerpolynom $N(s)$ der Übertragungsfunktion 
\[  
    G(s) 
        = \frac{Z(s)}{N(s)} 
        = \frac{Z(s)}{a_0s^0 + a_1s^1 + a_2s^2 + \dots + a_ns^n} 
\]
die folgenden Kriterien erfüllt.
%
\begin{enumerate}
    \item Alle Koeffizenten des Polynoms für den gegebenen Grad sind 
        vorhanden. \[ a_n \neq 0 \]
    \item Alle Koeffizienten des Polynoms sind grösser als null (also
        positiv). \[ a_n > 0 \]
    \item Die Determinante und alle Unterdeterminanten sind grösser 
        als null. \[ \Delta_i > 0 \]
\end{enumerate}
%
Für Polynome vom Grad $n \leq 2$ genügt die Prüfung der ersten beiden 
Kriterien.

\subsubsection{Determinanten berechnen}
Die Determinante eines Polynoms 
\[  
    N(s) =  a_0s^0 + a_1s^1 + a_2s^2 + \dots + a_ns^n
\]
lautet
\[ 
    \Delta_n =
       \begin{vmatrix}
            a_1 & a_3 & a_5 & a_7 & \dots \\
            a_0 & a_2 & a_4 & a_6 & \dots \\
            0   & a_1 & a_3 & a_5 & \dots \\
            0   & a_0 & a_2 & a_4 & \dots \\
            0   & 0   & a_1 & a_3 & \dots \\
            0   & 0   & a_0 & a_2 & \dots \\
            \vdots & \vdots & \vdots & \vdots & \ddots \\
        \end{vmatrix}
\]
Die Wahl der Kantengrösse hängt vom Grad des betrachteten Polynoms ab.
Hat man ein Polynom von Grad $n$, dann muss die Matrix auch $n$ Zeilen
und $n$ Spalten haben. 

Die Unterdeterminanten dieser Determinante sind dann all jene Matrizen,
welche eine kleinere Kantenlänge haben. Diese sind allesamt ausgehend
von der linken oberen Ecke anzusetzen.
\[  
    \Delta_1 =
        \begin{vmatrix}
            a_1 \\
        \end{vmatrix}
\]
\[  
    \Delta_2 =
        \begin{vmatrix}
            a_1 & a_3 \\
            a_0 & a_2 \\
        \end{vmatrix}
\]
\[  
    \Delta_3 =
        \begin{vmatrix}
            a_1 & a_3 & a_5 \\
            a_0 & a_2 & a_4 \\
            0   & a_1 & a_3 \\
        \end{vmatrix}
\]
Die Berechnung der Determinaten kann mittels zweier Sätze erfolgen:
\begin{itemize}
    \item Laplace'scher Entwicklungssatz (rekursiv)
    \item Regel von Sarrus
\end{itemize}

\subsubsection{Laplace'scher Entwicklungssatz}
Mit dem Laplace'schen Entwicklungssatz lassen sich Determinanten
rekursiv berechnen. Hierzu kann nach Spalten oder Zeilen vorgegangen
werden. Das folgende Beispiel zeigt das Verahren nach Zeilen. 

Man nimmt das erste Element $a$ und multipliziert es mit der Determinaten
der Untermatrix. Dann subtrahiert man das Produkt des nächsten Elements 
$b$ und der Determinante von dessen Untermatrix usw. 

Bei diesem Vorgang ist zu beachten, dass das Vorzeichen zwischen den
Summanden alterniert. 
\[  
    det(A)
        =   \begin{vmatrix}
                a & b & c \\
                d & e & f \\
                g & h & i \\
            \end{vmatrix}
        = a \cdot 
            \begin{vmatrix} 
                e & f \\
                h & i \\
            \end{vmatrix}
            - 
            d \cdot 
            \begin{vmatrix}
                b & c \\
                h & i \\
            \end{vmatrix}
            + 
            g \cdot 
            \begin{vmatrix}
                b & c \\
                e & f \\
            \end{vmatrix} 
\]

\subsubsection{Regel von Sarrus}
Die Regel von Sarrus ermöglicht es die Determinate einer Matrix direkt
auszudrücken. Hierzu wird jeweils diagonal durch die Matrix ein Produkt
aufgestellt. Alle Produkte gleichgerichteter Diagonalen werden 
aufsummiert, wobei die eine Richtung positiv und die andere negativ
belegt wird.
\[  
    det(A)
        =   \begin{vmatrix}
                a & b & c \\
                d & e & f \\
                g & h & i \\
            \end{vmatrix}
        = (aei + dhc + gbf) - (ceg + fha + ibd)
\]

\subsection{Nyquist}
Mit dem Nyquist-Kriterium wird die Stabilität eines Regelkreises anhand des 
offenen Kreises untersucht. 
\begin{itemize}
    \item spezielles Nyquist-Kriterium (offener Kreis stabil und maximal 
        2 Pole mit einem Realteil von $0$)
    \item allgemeines Nyquist-Kriterium (offener Kreis darf instabil sein)
\end{itemize}

\subsubsection{Spezielles Nyquist-Kriterium}
Das spezielle Nyquist-Kriterium untersucht die Stabilität anhand der
Phasenreserve des offenen Regelkreises. 
Hierzu wird einfach die Rückführung getrennt. So wird die 
Übertragungsfunktion zu
\[  
    G(s) = \frac{G_0}{1+G_0} \Rightarrow G_{\text{offen}}(s) = G_0
\]
%
\begin{figure}[h!]
    \begin{signalflow}[node distance=15mm]
        % Elemente
        \tikzgrid{
            \node[input]    (in)    {}          &
            \node[adder]    (a1)    {}          &
            \node[delay]    (d1)    {$G_0$}     &
            \node[node]     (n1)    {}          &
            \node[output]   (out)   {}
        }
        % Pfade
        \path[r>] (in) -- (a1);
        \path[r>] (a1) -- (d1);
        \path[r>] (d1) -- (out);
        \coordinate (p1) at ($ (n1) + (-1,-0.5) $);
        \coordinate (p2) at ($ (p1) + (-0.2,0) $);
        \draw[thick] (n1) |- (p1) -- ++(-0.2,-0.2);
        \path[r>] (p2) -| node[left, pos=0.95] {$-$} (a1);
    \end{signalflow}
\end{figure}
%

\subsubsection{Allgemeines Nyquist-Kriterium}
Das allgemeine Nyquist-Kriterium prüft die Stabilität des geschlossenen
Regekreises anhand der Pole des offenen Kreises. 
%
% \begin{figure}[h!]
%     \begin{signalflow}[node distance=15mm]
%         % Elemente
%         \tikzgrid{
%             \node[input]    (in)    {}          &
%             \node[adder]    (a1)    {}          &
%             \node[delay]    (d1)    {$G_0$}     &
%             \node[node]     (n1)    {}          &
%             \node[output]   (out)   {}
%         }
%         % Pfade
%         \path[r>] (in) -- (a1);
%         \path[r>] (a1) -- (d1);
%         \path[r>] (d1) -- (out);
%         \path[r>] (n1) -- ++(0,-0.5) -| node[left, pos=0.95] {$-$} (a1);
%     \end{signalflow}
% \end{figure}
%
Um dies zu prüfen, müssen zuerst die 
Parameter $a$ und $r$ bestimmt werden. $a$ bezeichnet die Anzahl der Pole
auf der imaginären Achse und $r$ die Anzahl der Pole rechtsseitig der
imaginären Achse. Ist dann die folgende Bedingung 
\[  
    \Delta \varphi = a \frac{\pi}{2} + r \pi
\]
mit diesen Parametern erfüllt, wobei $\Delta \varphi$ den resultierenden 
Winkel bezeichnet, dann ist $G_0$ stabil.

\subsection{Phasenreserve}
\begin{figure}[h!]
    \centering
    \begin{tikzpicture}[scale=2]
        \draw[->, name path=real] (-1.5,0) -- (1.5,0) node[below] {$Re$};
        \draw[->] (0,-1.5) -- (0,1.5) node[left] {$Im$};
        \draw[thick] 
            (-1,-0.1) node[below left] {$-1$} -- (-1,0.1);
        \draw[dotted, name path=kreis] (0,0) circle (1);
        \draw[name path=kurve] (-1.5,-1.5) 
            to [out=20, in=160] (-0.2,0.2)
            to [out=-20, in=90] (0,0);
        \draw[name intersections={of=real and kurve, name=amp}]
            (amp-1) node[above left] {$\omega = \omega_\pi$} -- ++(0,0.5);
        \draw[<->]
            (amp-1) +(0,0.4) -- node[above] {$\dfrac{1}{A_R}$} (0,0.4);
        \draw[name intersections={of=kreis and kurve, name=arc}]
            (arc-1) node[left] {$\omega = \omega_D$} -- (0,0);
        \draw[->] (-0.5,0) node[below right] {$\varphi_R$} arc [start angle=180, end angle=210, radius=0.5];
    \end{tikzpicture}
\end{figure}
\[
    \omega = \omega_D
\]
\[
    |G_0(j \omega_D)| = 1
\]
\[
    \varphi_R = \angle G_0(j \omega_D) + \pi
\]
Gutes Regelverhalten: $\varphi_R \geq 60^\circ$ \\
Gutes Störverhalten:  $\varphi_R \geq 30^\circ$

\subsubsection{Bode-Diagramm}
Die Phasenreserve kann auch im Bode-Diagramm ausgelesen werden. Sie 
beschreibt den Winkelunterschied von $-180^\circ$ und den Winkel beim
$0$dB Punkt.
%
\begin{figure}[h!]
    \centering
    \begin{tikzpicture}
        % Coordinaten
        \coordinate (a1) at (0,0);
        \coordinate (a2) at (0,-3);
        % Amplitudengang
        \begin{scope}[shift={(a1)}]
            % Koordinatensystem 
            \draw[->] (-0.1,0) -- (6.5,0) node[below] {$\omega$};
            \draw[->] (0,-2.0) -- (0,1) node[left] 
                {$20 \log \left( \left\lvert G \right\rvert \right)$};
            % x-Tics
            \draw[] (1,0.1) -- (1,-0.1);
            \draw[] (2,0.1) -- (2,-0.1);
            \draw[] (3,0.1) -- (3,-0.1);
            \draw[] (4,0.1) -- (4,-0.1);
            \draw[] (5,0.1) -- (5,-0.1);
            % y-Tics
            \draw[] (0.1,0.5) -- (-0.1,0.5);
            \draw[] (0.1,0.0) -- (-0.1,0.0) node[left] {$0$dB};
            \draw[] (0.1,-0.5) -- (-0.1,-0.5);
            \draw[] (0.1,-1.0) -- (-0.1,-1.0);
            \draw[] (0.1,-1.5) -- (-0.1,-1.5);
            % Amplitudengang
            \draw[red, thick] (0,0.5) -- (2,0.5) -- (6,-1.5);
        \end{scope}
        % Phasengang
        \begin{scope}[shift={(a2)}]
            % Koordinatensystem 
            \draw[->] (-0.1,0) -- (6.5,0) node[below] {$\omega$};
            \draw[->] (0,-3) -- (0,0.5) node[left] {$\angle\varphi$};
            % x-Tics
            \draw[] (1,0.1) -- (1,-0.1);
            \draw[] (2,0.1) -- (2,-0.1);
            \draw[] (3,0.1) -- (3,-0.1);
            \draw[] (4,0.1) -- (4,-0.1);
            \draw[] (5,0.1) -- (5,-0.1);
            % y-Tics
            \draw[] (0.1,0) -- (-0.1,0) node[left] {$0^\circ$};
            \draw[] (0.1,-0.5) -- (-0.1,-0.5) node[left] {$-45^\circ$};
            \draw[] (0.1,-1.0) -- (-0.1,-1.0) node[left] {$-90^\circ$};
            \draw[] (0.1,-1.5) -- (-0.1,-1.5) node[left] {$-135^\circ$};
            \draw[] (0.1,-2.0) -- (-0.1,-2.0) node[left] {$-180^\circ$};
            \draw[] (0.1,-2.5) -- (-0.1,-2.5) node[left] {$-225^\circ$};
            % Amplitudengang
            \draw[red, thick] (0,0) -- (1,0) -- (6,-2.5);
        \end{scope}
        \draw[dotted, blue] (0,-5) -- (6,-5);
        \draw[dotted, blue] (3,-5) -- (3,0);
        \draw[blue, thick, <->] (3,-5) -- (3,-4) 
            node[left, midway] {$\varphi_R$};
    \end{tikzpicture}
\end{figure}
%


\subsection{Amplitudenreserve}
Wie die Phasenreserve kann auch die Amplitudenreserve aus dem Bode-Diagramm 
ausgelesen werden für den offenen Regelkreis. Die Amplitudenreserve
beschreibt den Abstand vom $0$dB Punkt bis zum Punkt der 
Amplitudenverstärkung bei einer Phase von $-180^\circ$.
%
\begin{figure}[h!]
    \centering
    \begin{tikzpicture}
        % Coordinaten
        \coordinate (a1) at (0,0);
        \coordinate (a2) at (0,-3);
        % Amplitudengang
        \begin{scope}[shift={(a1)}]
            % Koordinatensystem 
            \draw[->] (-0.1,0) -- (6.5,0) node[below] {$\omega$};
            \draw[->] (0,-2.0) -- (0,1) node[left] 
                {$20 \log \left( \left\lvert G \right\rvert \right)$};
            % x-Tics
            \draw[] (1,0.1) -- (1,-0.1);
            \draw[] (2,0.1) -- (2,-0.1);
            \draw[] (3,0.1) -- (3,-0.1);
            \draw[] (4,0.1) -- (4,-0.1);
            \draw[] (5,0.1) -- (5,-0.1);
            % y-Tics
            \draw[] (0.1,0.5) -- (-0.1,0.5);
            \draw[] (0.1,0.0) -- (-0.1,0.0)node[left] {$0$dB};
            \draw[] (0.1,-0.5) -- (-0.1,-0.5);
            \draw[] (0.1,-1.0) -- (-0.1,-1.0);
            \draw[] (0.1,-1.5) -- (-0.1,-1.5);
            % Amplitudengang
            \draw[red, thick] (0,0.5) -- (2,0.5) -- (6,-1.5);
        \end{scope}
        % Phasengang
        \begin{scope}[shift={(a2)}]
            % Koordinatensystem 
            \draw[->] (-0.1,0) -- (6.5,0) node[below] {$\omega$};
            \draw[->] (0,-3) -- (0,0.5) node[left] {$\angle\varphi$};
            % x-Tics
            \draw[] (1,0.1) -- (1,-0.1);
            \draw[] (2,0.1) -- (2,-0.1);
            \draw[] (3,0.1) -- (3,-0.1);
            \draw[] (4,0.1) -- (4,-0.1);
            \draw[] (5,0.1) -- (5,-0.1);
            % y-Tics
            \draw[] (0.1,0) -- (-0.1,0) node[left] {$0^\circ$};
            \draw[] (0.1,-0.5) -- (-0.1,-0.5) node[left] {$-45^\circ$};
            \draw[] (0.1,-1.0) -- (-0.1,-1.0) node[left] {$-90^\circ$};
            \draw[] (0.1,-1.5) -- (-0.1,-1.5) node[left] {$-135^\circ$};
            \draw[] (0.1,-2.0) -- (-0.1,-2.0) node[left] {$-180^\circ$};
            \draw[] (0.1,-2.5) -- (-0.1,-2.5) node[left] {$-225^\circ$};
            % Amplitudengang
            \draw[red, thick] (0,0) -- (1,0) -- (6,-2.5);
        \end{scope}
        \draw[dotted, blue] (5,-5) -- (5,0);
        \draw[dotted, blue] (0,-5) -- (5,-5);
        \draw[blue, thick, <->] (5,0) -- (5,-1) 
            node[right, midway] {$A_R$};
    \end{tikzpicture}
\end{figure}
%

\[
    \omega = \omega_\pi
\]
\[
    Im(G_0(j \omega_\pi)) = 0 \quad \to \omega_{\pi_1}, \omega_ {\pi_2}, \ldots
\]
\[
    A_R = \frac{1}{|G_0(j \omega_R)|}
\]
Gutes Regelverhalten: $A_R \geq 2$

\subsection{Totzeitreserve}
\[
    T_t = \frac{\varphi_R}{\omega_d}
\]

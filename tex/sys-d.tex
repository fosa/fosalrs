% coding:utf-8

%FOSALRS, a LaTeX-Code for a electrical summary of control theory
%Copyright (C) 2013, Daniel Winz, Ervin Mazlagic

%This program is free software; you can redistribute it and/or
%modify it under the terms of the GNU General Public License
%as published by the Free Software Foundation; either version 2
%of the License, or (at your option) any later version.

%This program is distributed in the hope that it will be useful,
%but WITHOUT ANY WARRANTY; without even the implied warranty of
%MERCHANTABILITY or FITNESS FOR A PARTICULAR PURPOSE.  See the
%GNU General Public License for more details.
%----------------------------------------

\subsection{$D$}
DGL: 
\[
    x_a(t) = K_D \cdot \dot{x}_e(t) = T_D \cdot \dot{x}_e(t)
\]
Übertragungsfunktion: 
\[
    G(s) = K_D \cdot s = T_D \cdot s
\]
Sprungantwort: 
\[
    h(t) = K_D \cdot \delta(t) = T_D \cdot \delta(t)
\]
\begin{figure}[h!]
    \begin{tikzpicture}
        \draw[->] (-0.2,0) -- (5,0);
        \draw[->] (0,-0.2) -- (0,2.2);
        \draw[thick, red] (-0.2,0) -- (0,0);
        \draw[->, thick, red] (0,0) -- (0,1.5);
    \end{tikzpicture}
\end{figure}
\FloatBarrier
\noindent
Pol-Nullstellen-Plan: 
\begin{figure}[h!]
    \begin{tikzpicture}
        \draw[->] (-2,0) -- (2,0) node[above right] {$Re$};
        \draw[->] (0,-2) -- (0,2) node[above right] {$Im$};
        \node[] at (0,0) {O};
    \end{tikzpicture}
\end{figure}
\clearpage
\noindent
Bode-Diagramm: 
\begin{figure}[h!]
    \begin{tikzpicture}
        \draw[->] (-0.2,0) -- (5.2,0) node[above right] {$\omega$};
        \draw[->] (0,-2) -- (0,2.2) node[above right] {$|g|$};
        \draw[thick, red] (0,-1) -- (5,1);
    \end{tikzpicture}
\end{figure}
\begin{figure}[h!]
    \begin{tikzpicture}
        \draw[->] (-0.2,0) -- (5.2,0) node[above right] {$\omega$};
        \draw[->] (0,-2) -- (0,1.2) node[above right] {$\varphi$};
        \draw[thick, red] (0,1) node[left] {$90^\circ$} -- (5,1);
    \end{tikzpicture}
\end{figure}
\FloatBarrier
\noindent
Ortskurve: 
\begin{figure}[h!]
    \begin{tikzpicture}
        \draw[->] (-2,0) -- (2,0) node[above right] {$Re$};
        \draw[->] (0,-2) -- (0,2) node[above right] {$Im$};
        \draw[->, thick, red] (0,0) -- (0,1.5);
    \end{tikzpicture}
\end{figure}
\clearpage

% coding:utf-8

%FOSALRS, a LaTeX-Code for a electrical summary of control theory
%Copyright (C) 2013, Daniel Winz, Ervin Mazlagic

%This program is free software; you can redistribute it and/or
%modify it under the terms of the GNU General Public License
%as published by the Free Software Foundation; either version 2
%of the License, or (at your option) any later version.

%This program is distributed in the hope that it will be useful,
%but WITHOUT ANY WARRANTY; without even the implied warranty of
%MERCHANTABILITY or FITNESS FOR A PARTICULAR PURPOSE.  See the
%GNU General Public License for more details.
%----------------------------------------

\subsection{$IT_1$}
DGL: 
\[
    T_1 \cdot \dot{x}_a + x_a(t)
    = K_I \cdot \int\limits_{0}^{t }x_e(t) ~ dt + x_a(0)
    = frac{x_{aN}}{x_{eN}} \cdot T_I \cdot \int\limits_{0}^{t }x_e(t) ~ dt + x_a(0)
\]
Übertragungsfunktion: 
\[
    G(s) = \frac{K_I}{s(1 + T_1 \cdot s)}
\]
Sprungantwort: 
\[
    h(t) = K_I \cdot \left(t - T_1\left(1 - e^{-\frac{t}{T_1}}\right)\right)
\]
\begin{figure}[h!]
    \begin{tikzpicture}
        \draw[->] (-0.2,0) -- (5,0);
        \draw[->] (0,-0.2) -- (0,2.2);
        \draw[thick, red] (-0.2,0) -- (0,0) to[out=0, in=-135] (2,1) -- (3,2);
    \end{tikzpicture}
\end{figure}
\FloatBarrier
\noindent
Pol-Nullstellen-Plan: 
\begin{figure}[h!]
    \begin{tikzpicture}
        \draw[->] (-2,0) -- (2,0) node[above right] {$Re$};
        \draw[->] (0,-2) -- (0,2) node[above right] {$Im$};
        \node[] at (-1.5,0) {X};
        \node[below] at (-1.5,0) {$-\frac{1}{T_1}$};
        \node[] at (0,0) {X};
        \node[below] at (-1.5,0) {$-\frac{1}{T_1}$};
    \end{tikzpicture}
\end{figure}
\clearpage
\noindent
Bode-Diagramm: 
\begin{figure}[h!]
    \begin{tikzpicture}
        \draw[->] (-0.2,0) -- (5.2,0) node[above right] {$\omega$};
        \draw[->] (0,-2) -- (0,1.2) node[above right] {$|g|$};
        \draw[thick, red] (0,0.5) node[left] {$K_{P_{dB}}$} -- (2.5,-0.5) -- (5,-2.5);
    \end{tikzpicture}
\end{figure}
\begin{figure}[h!]
    \begin{tikzpicture}
        \draw[->] (-0.2,0) -- (5.2,0) node[above right] {$\omega$};
        \draw[->] (0,-2) -- (0,1.2) node[above right] {$\varphi$};
        \draw[thick, red] (0,-1) -- (1.5,-1) -- (3.5,-2) -- (5,-2);
        \draw[-] (-0.2,-1) node[left] {$-90^\circ$} -- (0.2,-1);
        \draw[-] (-0.2,-2) node[left] {$-180^\circ$} -- (0.2,-2);
    \end{tikzpicture}
\end{figure}
\FloatBarrier
\noindent
Ortskurve: 
\begin{figure}[h!]
    \begin{tikzpicture}
        \draw[->] (-2,0) -- (2,0) node[above right] {$Re$};
        \draw[->] (0,-2) -- (0,2) node[above right] {$Im$};
        \draw[->, thick, red] (-1.5,-2) to[out=90, in=-90] (-1.5,-1) to[out=90,in=180] (0,0);
        \draw[] (-1.5,0.2) node[above] {$-\frac{T_1}{T_I}$} -- (-1.5,-0.2);
    \end{tikzpicture}
\end{figure}
\clearpage
